\section{Conclusion}
Ainsi il a été possible d'obtenir les pulsations et frottements d'un oscillateur libre. Connaissant la théorie cela permet de trouver sa pulsation dans le cadre idéal sans frottement mais également à partir de ces constantes d'effectuer d'autres analyses comme celles du déphasage et de la résonance. Les mesures prises pour ces cas ont bien correspondu à celles obtenues par la théorie et les valeurs calculées pour les oscillations libres. De plus, le phénomène de battement a également pu être observé et caractérisé. Ainsi le système présenté ici semble être cohérent avec la théorie solide sur les oscillateurs harmoniques ce qui indique la bonne qualité du montage. Les phénomènes de résonance ont également pu être caractérisé ce qui peut avoir ensuite de nombreuses applications en comparant un système similaire à cet oscillateur qu'il est possible d'étudier précisément.