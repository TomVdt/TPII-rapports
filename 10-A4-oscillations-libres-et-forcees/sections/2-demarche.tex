\section{Démarche Expérimentale}
Afin de réaliser ces mesures le montage utilisé était tel que présenté en \autoref{fig:montage_exp}. L'angle de rotation $\theta$ du disque ainsi que du moteur étaient mesurés par un plotteur X-Y prenant un voltage étalonné arbitrairement. En effet, les mesures d'intérêt concerne ici la variation de l'amplitude et non sa valeur numérique exacte. Le moment d'inertie du disque pouvait également être modifié en rajoutant des masses équilibrées autour de son axe. Les intensités d'amortissement magnétique étaient elles aussi arbitraires et seront exprimées par la suite par un pourcentage de l'amortissement maximal disponible dans le montage.

Pour étudier les oscillations libres trois moments d'inertie $I$ différents ont été étudiés, correspondant à 0, 2 et 3 masses rajoutées sur le disque. Pour chaque $I$, 5 valeurs d'amortissement différentes ont été testées. Les pulsations ont été relevées par extraction de points sur les courbes. Pour les battements, plusieurs essais ont été effectués afin de trouver empiriquement des conditions initiales permettant une bonne observation du phénomène. Pour étudier la résonance finalement il a été nécessaire de relever plusieurs fois des oscillations en régime permanent pour différentes valeurs de $\Omega$ la pulsation du moteur. Il a ensuite été possible de mesurer l'amplitudes et le déphasage pour chacun de ces relevés. Pour le battement et la résonance les pulsations ont été étudiées à l'aide d'une transformation de Fourier.