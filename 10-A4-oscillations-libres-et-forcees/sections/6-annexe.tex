\section{Calcul d'erreurs}
\label{sec:erreurs}

Les erreurs sur les mesures sont données dans le \autoref{tab:erreurs}.

\begin{table}[h]
    \centering
    \begin{tabulary}{\textwidth}{C C}
        \toprule
        Variable & Erreur \\
        \midrule
        Temps \(t\) [\si{\second}] & 0.05 \\
        \bottomrule
    \end{tabulary}
    \caption{Erreurs estimées sur les mesures}
    \label{tab:erreurs}
\end{table}

\paragraph*{Regression linéaire}
Les erreurs sur les régressions linéaires \(y = ax + b\) sur les mesures \((x_i, y_i) ; i = \{1, \dots, n\}\) sont donnés par \cite{erreursmesure}:

\begin{equation}
    \label{eq:erreur:fit}
    \begin{aligned}
        (\Delta a)^2 &= \frac{\sum_{i=1}^{n}(y_i - (a x_i + b))^2}{(n-2) \sum_{i=1}^{n}(x_i - \bar{x})^2}\\
        \Delta b &= \bar{x} \Delta a + a \Delta \bar{x}
    \end{aligned}
\end{equation}

En pratique, ces valeurs sont calculées par la bibliothèque python \texttt{numpy}.

\paragraph*{Formules d'erreurs}

Erreur sur la pulsation \(\omega\) et \(\Omega\):
\begin{equation}
    \Delta (\omega) = \frac{2 \pi}{T^2} \Delta (T), \qquad
    \Delta (\Omega) = \frac{2 \pi}{T^2} \Delta (T)
\end{equation}
où l'erreur sur la periode \(T\) est trouvée à partir de la mesures de \(n\) periodes par:
\begin{equation}
    \Delta (T) = \frac{\Delta (t)}{n}
\end{equation}

Erreur sur la pulsation de l'oscillateur idéal \(\omega_0\):
\begin{equation}
    \Delta (\omega_0) = \frac{\omega \Delta (\omega) + \lambda \Delta (\lambda)}{\omega_0}
\end{equation}

Erreur sur la pulsation de battement \(\omega_B\):
\begin{equation}
    \Delta (\omega_B) = \Delta (\omega) + \Delta (\Omega)
\end{equation}

Erreur sur la largeur de la raie $\Delta \Omega$:
\begin{equation}
    \Delta (\Delta \Omega) = \absfrac{\Delta \Omega}{\lambda}\Delta (\lambda) + \absfrac{\Delta \Omega}{\omega} \Delta(\omega) + \absfrac{\Delta \Omega}{\Omega_r}\Delta(\Omega)
\end{equation}

Erreur sur le facteur qualité $Q$:
\begin{equation}
    \Delta (Q) = \absfrac{Q}{\Omega_r} \Delta (\Omega) + \absfrac{Q}{\Delta \Omega}\Delta(\Delta \Omega)
\end{equation}

Toutes ces erreurs sont calculées en pratique par la bibliothèque \texttt{uncertainties}.
