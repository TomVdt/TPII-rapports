\section{Résultats}

\paragraph{Oscillations libres}
Le coefficient d'amortissement et la pulsation du disque ont été déterminé pour des amortissements et masses différentes. Le disque a été laché avec un angle initial de 135°. Conformement à l'\autoref{eq:oscil_libre}, une régression exponentielle sur l'enveloppe du signal a été réalisée afin d'obtenir le coefficient d'amortissement \(\lambda\). Un des signaux obtenus lors de l'expérience est présenté en \autoref{fig:signal_libre}. Le coefficient \(\lambda\) a été déterminé pour 0, 2, ou 3 masses supplémentaires ainsi que pour un amortissment variant entre 0 et 100\%. Les résultats obtenus sont présentés en \autoref{fig:lambda_libre}. Le calcul de la periode \(T\) a été réalisée en prenant une moyenne sur plusieurs periodes, en fonction du nombre de periodes disponnibles, permettant d'obtenir \(\omega = 2 \pi / T\). Cela résulte en une plus grande incertitude sur \(\omega\) pour des grands amortissement. Les détails sur l'estimation des incertitudes sont disponnibles en \autoref{sec:erreurs}. La \autoref{fig:omega_libre} présente les résultats obtenus.

\begin{figure}[h]
    \centering
    \includegraphics[width=0.98\linewidth]{figures/I3_50_nomot_fitted.pdf}
    \caption{Oscillation libre du disque avec 3 masses supplémentaires et un amortissement de 50\%}
    \label{fig:signal_libre}
\end{figure}

\begin{figure}[h]
    \centering
    \begin{subfigure}{0.48\linewidth}
        \centering
        \includegraphics[width=\linewidth]{figures/lambda_nomot.pdf}
        \caption{}
        \label{fig:lambda_libre}
    \end{subfigure}
    \begin{subfigure}{0.48\linewidth}
        \centering
        \includegraphics[width=\linewidth]{figures/omega_nomot.pdf}
        \caption{}
        \label{fig:omega_libre}
    \end{subfigure}
    \caption{(a) Le coefficient d'amortissement \(\lambda\) et (b) la pulsation \(\omega\) pour différentes configurations de masses supplémentaires, en fonction de l'amortissement appliqué}
\end{figure}

La pulsation \(\omega_0\) de l'oscillateur harmonique libre sans frottements peut alors être détérminé à partir de l'\autoref{eq:omega_0}. Puisque \(\omega_0\) est censé être constant pour les différents amortissements appliqués, la valeur moyenne a été retenue. Les résultats sont présentés dans le \autoref{tab:omega0_libre}.

\begin{table}[h]
    \centering
    \begin{tabulary}{0.9\linewidth}{C C C C}
        \toprule
        & 0 masses & 2 masses & 3 masses \\
        \midrule
        \(\omega_0\) [\si{\radian\per\second}] & \(4.03 \pm 0.02\) & \(3.14 \pm 0.02\) & \(2.87 \pm 0.02\) \\
        \bottomrule
    \end{tabulary}
    \caption{Valeurs moyennes de la pulsation de l'oscillateur harmonique sans frottements}
    \label{tab:omega0_libre}
\end{table}

\paragraph{Battements transitoires}
Pour les battements dans la solution transitoire de l'\autoref{eq:oscillations_forcees} plusieurs essais de conditions initiales ont été essayés pour le disque à 0 masses et amortissement de 50\%. Le résultat permettant de mieux visualiser les battements est présenté en \autoref{fig:bat_ampl}. Les amplitudes affichées ont été obtenues en prélevant les valeurs absolues des pics minimums et maximums des oscillations. En prenant l'écart entre les deux minimums d'amplitude à $t_1$ et $t_2$ il est possible d'obtenir une période de battement donnant la pulsation de battement expérimentale: $\omega_{B,exp} = (0.426 \pm 0.001)
$ \si{\radian\per\second}.
\begin{figure}[h]
    \centering
    \includegraphics[width=0.7\linewidth]{figures/bat_ampl.pdf}
    \caption{Oscillation forcées du disque et amplitudes avec condition initiale $\theta \gg 1$, 0 masses et un amortissement de 50\%, intervalle d'un battement}
    \label{fig:bat_ampl}
\end{figure}

La transformation de Fourier des signaux du disque et du moteur ont ensuite été prises afin d'obtenir leurs pulsations respectives. Les fonctions obtenues sont présentées en \autoref{fig:bat_fourier}. Deux pics symétriques sont obtenus correspondant à la fréquence $f$ et à $-f$. Cette valeur de $f$ pour les deux signaux permet d'obtenir les pulsations, pour le moteur \hbox{$\Omega = (3.47\pm0.10)$ \si{\radian\per\second}} et pour les oscillations $\omega = 3.84\pm0.12$ \si{\radian\per\second}. Cela donne en utilisant la formule \hbox{\(\omega_{B,th} = |\omega- \Omega| = 0.4\pm0.2\)}. L'écart relatif entre $\omega_{B,exp}$ et $\omega_{B,th}$ par rapport à $\omega_{B,th}$ est donc de 10\%.
\begin{wrapfigure}{R}{0.5\linewidth}
    % \vspace*{-0.5cm}
    \centering
    \includegraphics[width=\linewidth]{figures/bat_fourier.pdf}
    \caption{Transformation de Fourier des oscillations du disque et du moteur}
    \label{fig:bat_fourier}
    %%%%% WARNING %%%%%%%
    \vspace*{-0.5cm}
\end{wrapfigure}


\paragraph{Résonance}


