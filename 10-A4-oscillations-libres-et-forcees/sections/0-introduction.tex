\section{Introduction}

Dans le domaine de la construction, l'étude des propriétés oscillantes d'une structure est essentielle afin de permettre une longue vie. Par exemple, l'étude de la résonnance dans des structures comme des ponts est vitale à sa sécurité. Le \textit{Millennium Bridge} à Londre est un exemple connu du phénomène de résonnance. À son ouverture en 2000, en raison de la fréquence de marche des piétons, le pont s'est mis à osciller, ce qui a necessité d'importants travaux d'ajustements. Après le désastre du \textit{Tacoma Bridge}, l'étude de la résonnance entre structure et environnement est devenue essentielle \cite{autre-pont-foireux}. Cependant, dans le cas du \textit{Millennium Bridge}, ce cas précis de résonnance n'avait alors pas été étudié \cite{pont-foireux}.

Cette expérience repose dans un premier temps sur l'analyse du comportement d'un disque oscillant de façon libre, en déterminant le coefficient d'amortissement et la pulsation du disque. Ensuite, l'étude de l'amplitude et la phase lors d'oscillation forcées permetteront de trouver la fréquence de résonnance du disque. Enfin, le phénomène de battements sera étudié.
