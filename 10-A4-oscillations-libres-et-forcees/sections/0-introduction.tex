\section{Introduction}

Dans le domaine de la construction, l'étude des propriétés oscillatoires d'une structure est essentielle afin de permettre une longue durée de vie du bâtiment. Par exemple, l'étude de la résonance dans des structures comme des ponts est vitale à leur sécurité. Le \textit{Millennium Bridge} à Londre est un exemple connu du phénomène de résonance. À son ouverture en 2000, en raison de la fréquence de marche des piétons, le pont s'est mis à osciller, ce qui a necessité d'importants travaux d'ajustements. Après le désastre du \textit{Tacoma Bridge}, l'étude de la résonance entre structure et environnement est devenue essentielle \cite{autre-pont-foireux}. Cependant, dans le cas du \textit{Millennium Bridge}, ce cas précis de résonance n'avait alors pas été étudié \cite{pont-foireux}.

Cette expérience repose dans un premier temps sur l'analyse du comportement d'un disque oscillant de façon libre, en déterminant le coefficient d'amortissement et la pulsation du disque. Ensuite, le phénomène de battements dans la solution transitoire sera étudié. Enfin, l'étude de l'amplitude et la phase lors d'oscillation forcées permetteront de trouver la fréquence de résonnance du disque.
