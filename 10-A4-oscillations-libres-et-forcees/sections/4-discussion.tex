\section{Discussion}

\paragraph{Oscillations libres}
Les résultats obtenus montrent que la masse a un effet important sur l'amortissement du disque. En effet, la \autoref{fig:lambda_libre} montre que le coefficient d'amortissement \(\lambda\) grandit 2 fois moins vite pour le disque avec 3 masses que pour le disque avec aucune masse. La pulsation \(\omega\) du disque reste cependant plus ou moins constante, ce qui peut s'expliquer par le faible coefficient d'amortissement \(\lambda\). Les valeurs obtenues pour \(\omega_0\) avaient une différence maximale de \((2 \pm 0.1) \cdot 10^{-2}\) \si{\radian\per\second} (soit 0.6\% d'écart relatif avec \(\omega\)) pour le disque sans masses et \((0.7 \pm 0.1) \cdot 10^{-2}\) \si{\radian\per\second} (soit 0.25\% d'écart relatif avec \(\omega\)) pour le disque avec 3 masses supplémentaires. Il semble donc que les frottements n'ont que très peu d'effets sur la pulsation du disque.

Dans le cas du freinage électromagnétique nul, il est toujours possible d'obtenir un coefficient d'amortissment et une pulsation en raison des autres frottements présents: frottements avec l'air, frottements sur l'axe du disque ou encore des effets magnétiques residuels.

La valeur de \(\lambda\) avec une très grande incertitude dans la \autoref{fig:lambda_libre} (2 masses, 100\% d'amortissement) peut s'expliquer par une grande quantité de bruit dans la fin du signal ainsi qu'un faible temps d'oscillation, donnant moins de points pour la régression exponentielle sur l'enveloppe. L'augmentation de l'incertitude dans la \autoref{fig:omega_libre} est attendue en raison du nombre de periodes qu'il est possible de mesurer pour des grand amortissement.