\section{Discussion}

\paragraph{Oscillations libres}
Les résultats obtenus montrent que la masse a un effet important sur l'amortissement du disque. En effet, la \autoref{fig:lambda_libre} montre que le coefficient d'amortissement \(\lambda\) grandit 2 fois moins vite pour le disque avec 3 masses que pour le disque avec aucune masse. La pulsation \(\omega\) du disque reste cependant plus ou moins constante selon l'amortissements appliqué, ce qui peut s'expliquer par le faible coefficient d'amortissement \(\lambda\). La différence maximale entre \(\omega\) et \(\omega_0\) était de \((2 \pm 0.1) \cdot 10^{-2}\) \si{\radian\per\second} (soit 0.6\% d'écart relatif) pour le disque sans masses et \((0.7 \pm 0.1) \cdot 10^{-2}\) \si{\radian\per\second} (soit 0.25\% d'écart relatif) pour le disque avec 3 masses supplémentaires. Il semble donc que les frottements n'ont que très peu d'effets sur la pulsation du disque.

La valeur de \(\lambda\) avec une très grande incertitude dans la \autoref{fig:lambda_libre} (2 masses, 100\% d'amortissement) peut s'expliquer par une grande quantité de bruit dans la fin du signal ainsi qu'un faible temps d'oscillation, donnant moins de points pour la régression exponentielle sur l'enveloppe. L'augmentation de l'incertitude dans la \autoref{fig:omega_libre} est attendue en raison du faible nombre de periodes qu'il est possible de mesurer pour de grand amortissement.


\paragraph{Battements}
Les battements obtenus étaient pour la plupart peu prononcé il a donc fallu effectuer plusieurs essais pour illustrer correctement le phénomène. De plus, aucun des essais n'a permis de distinguer clairement plus d'un battement avant la prépondérance de la solution permanente. Afin d'observer plus de battements il conviendrait de trouver un système avec un amortissement $\lambda$ plus faible afin de préserver la solution transitoire plus longtemps avec peut-être un $\Omega$ plus proche de $\omega$ afin de diminuer $\omega_B$ et observer plus de battements dans le même temps. Malgré l'abscence de plusieurs battements afin d'améliorer la mesure la valeur obtenue en prélevant sur la courbe d'oscillation était éloignée de seulement 10\% de celle obtenue grâce à l'analyse de Fourier ce qui indique que cette méthode reste viable pour donner de bonnes approximations.

Pour l'analyse de Fourier, la transformée obtenue était comme attendue et montre la grande utilité de cette méthode qui permet d'extraire efficacement des composantes d'un signal. Il est possible d'observer un creux à la fréquence 0 qui correspond à la suppression de la moyenne du signal.

\paragraph{Résonance}
La courbe de $A(\Omega)$ était comme attendue avec une résonance bien marquée montrant la bonne cohérence de l'expérience globale. Cependant les déphasages mesurés, bien que pour la plupart respectant le comportement attendu illsutré par $\psi_\mathrm{th}$ en \autoref{fig:res_psi}, ont souvent donnés des valeurs aberrrantes.

Les nécessité d'effectuer de nombreuses mesures, 30 échantillons eux même d'environ 20 secondes, permet de limiter les erreurs sur le comportement global comme illustré par la bonne courbe de résonance en \autoref{fig:res_A} mais a pu introduire des erreurs systématiques ou de légères modifications dans le système au fur et à mesure des manipulations. Cela pourrait expliquer les valeurs aberrantes observée pour certains déphasage $\psi$.

