\section{Résultats}

\paragraph*{Relation entre \(P_\gamma\) et \(\frac{1}{d^2}\)}
La puissance totale sortant de la lampe \(P_{tot}\) est constante. Cette puissance est répartie également sur toute la surface de la sphère correspondante à chaque instant de la propagation. Ainsi
\[P_{surf} = \frac{P_{tot}}{4 \pi r^2} = \frac{P_{tot}}{4 \pi} \frac{1}{d^2} \]
donnant alors une relation linéaire entre \(P_\gamma = P_{surf} S\) et \(\frac{1}{d^2}\), où \(P_{surf}\) est la puissance reçue par unité de surface.
Cette relation linéaire, pour des grandes valeurs de d, est mise en évidence dans la figure \ref{plot:1}.

\begin{figure}
    \centering
    \includegraphics[width=0.6\textwidth, height=5cm]{plots/Pgamma(1d2).tikz}
    \caption{Relation inversement quadratique entre distance et puissance lumineuse}
    \label{plot:1}
\end{figure}

\paragraph*{Courbes caractéristiques \(I_r(U_D)\) et \(P(U_D)\)}
La mesure de l'intensité du courant inverse de la diode \(I_R\) en fonction de la tension \(U_D\) aux bornes d'une résistance \(R_C\) elle-même branchée aux bornes de la diode permet de trouver la puissance de la cellule. Le courant \(I_R\) en fonction de \(U_D\) pour les différentes technologies de cellule est représenté à la Fig. \ref{plot:2}, pour deux conditions d'éclairement différentes.

\begin{figure}
    \centering
    \begin{subfigure}[c]{0.45\linewidth}
        \centering
        \includegraphics[width=\textwidth, height=6cm]{plots/Ir(U),A.tikz}
        \caption{Amorphe}
        \label{plot:2a}
    \end{subfigure}
    \begin{subfigure}[c]{0.45\linewidth}
        \centering
        \includegraphics[width=\textwidth, height=6cm]{plots/Ir(U),MP.tikz}
        \caption{Mono- et polycristallin}
        \label{plot:2b}
    \end{subfigure}
    \caption{Caractéristique des cellules photovoltaïques}
    \label{plot:2}
\end{figure}

\paragraph*{Puissance et rendement max}
La puissance \(P\) fournie par la cellule est donnée par \(P = I_R U_D\). La valeur de \(P_{max}\) est obtenue en considérant le point extremum de la fonction \(P(U_D)\), illustré à la Fig. \ref{plot:3}. Les valeurs de \(P_{max}\) sous les différentes conditions sont données à la Fig. \ref{tab:pmax}

\begin{figure}
    \centering
    \begin{subfigure}[c]{0.45\linewidth}
        \centering
        \includegraphics[width=\linewidth, height=6cm]{plots/P(U),A.tikz}
        \caption{Amorphe}
        \label{plot:3a}        
    \end{subfigure}
    \begin{subfigure}[c]{0.45\linewidth}
        \centering
        \includegraphics[width=\linewidth, height=6cm]{plots/P(U),MP.tikz}
        \caption{Mono- et polycristallin}
        \label{plot:3b}        
    \end{subfigure}
    \caption{Puissance \(P\) fournie par les cellules photovoltaïques}
    \label{plot:3}
\end{figure}

\begin{figure}
    \centering
    \begin{tabulary}{0.7\linewidth}{C C C}
        \toprule
        Cellule & \(d_1\) & \(d_2\) \\
        \midrule
        Amorphe & \((0.08 \pm 0.07)\) \unit{\watt} & \((0.03 \pm 0.07)\) \unit{\watt} \\
        Monocristalline & \((0.069 \pm 0.007)\) \unit{\watt} & \((0.028 \pm 0.007)\) \unit{\watt} \\
        Polycristalline & \((0.033 \pm 0.005)\) \unit{\watt} & \((0.013 \pm 0.003)\) \unit{\watt} \\
        \bottomrule
    \end{tabulary}
    \caption{Valeurs de \(P_{max}\)}
    \label{tab:pmax}
\end{figure}

Les rendements maximaux sont donnés par \(\eta_{max} = \frac{P_{max}}{P_\gamma}\), avec \(P_\gamma = (1,49 \pm 0.06)\) \unit{\watt} pour \(d_1\) et \(P_\gamma = (0.62 \pm 0.03)\) \unit{\watt} pour \(d_2\). La valeurs des rendements max sont donnés à la Fig. \ref{tab:rendementmax}. Les rendements maximums pour les même cellules sont similaires aux deux distances différentes. 


\begin{figure}
    \centering
    \begin{tabulary}{0.7\linewidth}{C C C}
        \toprule
        Cellule & \(d_1\) & \(d_2\) \\
        \midrule
        Amorphe & \((5.0 \pm 0.3)\)\% & \((4.6 \pm 0.3)\)\% \\
        Monocristalline & \((4.6 \pm 0.7)\)\% & \((4.5 \pm 1.0)\)\% \\
        Polycristalline & \((2.2 \pm 0.4)\)\% & \((2.0 \pm 0.6)\)\% \\
        \bottomrule
    \end{tabulary}
    \caption{Valeurs de \(\eta_{max}\)}
    \label{tab:rendementmax}
\end{figure}


\paragraph*{Grandeur caractéristique moyenne \(i = I_\gamma/P_\gamma\)}
Le courant court-circuit \(I_\gamma\) est mesuré en enlevant la resistance variable \(R_C\) du circuit. Le courant \(I_\gamma\) varie fortement selon la distance \(d\), comme illustré à la Fig. \ref{plot:4}. \\
Une relation linéaire entre \(I_\gamma\) et \(P_\gamma\) peut être observée, et permet de trouver une grandeur caractéristique moyenne \(i = \frac{I_\gamma}{P_\gamma}\) pour chaque cellule photovoltaïque. La pente de la régression linéaire donne la valeur de \(i\) pour la cellule, les valeurs sont données dans la Fig. \ref{plot:5}

\begin{figure}
    \centering
    \includegraphics[width=0.6\linewidth, height=6cm]{plots/Igamma(d),AMP.tikz}
    \caption{Courant photovoltaïque pour différentes cellules}
    \label{plot:4}
\end{figure}

\begin{figure}
    \centering
    \begin{subfigure}[t]{0.45\linewidth}
        \centering
        \includegraphics[width=\textwidth, height=6cm]{plots/i(Pgamma),A.tikz}
        \caption{Amorphe}
        \label{plot:5a}
    \end{subfigure}
    \begin{subfigure}[t]{0.45\linewidth}
        \centering
        \includegraphics[width=\textwidth, height=6cm]{plots/i(Pgamma),MP.tikz}
        \caption{Mono- et polycristallin}
        \label{plot:5b}
    \end{subfigure}
    \caption{Grandeur caractéristique i pour différentes cellules}
    \label{plot:5}
\end{figure}

\paragraph*{Role du spectre lumineux}
En utilisant des filtres dont les caractéristique sont donnés à la Fig. \ref{tab:filters} en annexe, il est possible d'observer la réaction de la cellule à différents spectres de lumière. La figure \ref{plot:6} donne la valeur de \(i\) pour chaque filtre. Les mesures ont été réalisées avec \(d = (42.0 \pm 0.2)\) \unit{\centi\meter}. Les cellules ne réagissent que très peu à la lumière des filtres 25 et 87C, correspondant aux couleurs rouge et aux infrarouges, alors que la lumière jaune-orange des filtres 2B, 8 et 16 donnent des valeurs plus proches des valeurs en l'absence de filtre. La cellule amorphe est aussi beaucoup moins sensible aux couleurs différentes du jaune-orange que les cellules cristallines.

\begin{figure}[H]
    \centering
    \begin{subfigure}[t]{0.48\linewidth}
        \centering
        \includegraphics[width=\linewidth, height=6cm]{plots/i(filtre),A.tikz}
        \caption{Amorphe}
        \label{plot:6a}
    \end{subfigure}
    \begin{subfigure}[t]{0.48\linewidth}
        \centering
        \includegraphics[width=\textwidth, height=6cm]{plots/i(filtre),MP.tikz}
        \caption{Mono- et polycristallin}
        \label{plot:6b}
    \end{subfigure}
    \caption{Grandeur caractéristique i des cellules photovoltaïques pour différents spectres}
    \label{plot:6}
\end{figure}

\paragraph*{Role de l'angle d'incidence}
Le courant \(I_\gamma\) traversant la cellule dépend aussi de l'angle d'incidence des rayons lumineux. Les mesures rapportées dans la figure \ref{plot:7} ont été réalisées en faisant varier l'angle de la cellule par rapport à l'axe avec la lampe de \(0^{\circ}\) à \(90^{\circ}\). Un fit sinusoïdal a été appliqué.

\begin{figure}[H]
    \centering
    \includegraphics[width=0.6\textwidth, height=6cm]{plots/Igamma(theta),MP.tikz}
    \caption{Réaction des cellules cristallines au changement d'angle d'incidence.}
    \label{plot:7}
\end{figure}


% Présentation des résultats des mesures avec conditions dans lesquelles les mesures ont
%  été effectuées, calculs des grandeurs dérivées et des incertitudes.
% — Tableaux et graphiques avec légendes incorporés au texte, unités physiques, rectangles
% d’erreurs.
% — La section résultats est purement descriptive : elle liste clairement les résultats obtenus :
% graphiques/tableaux avec les paramètres utilisés pour leur obtention (un externe doit
% pouvoir reproduire toutes vos mesures avec votre rapport).
% — Si certains points ne sont pas pris en compte, pour une raison justifiée, cela doit aussi
% être précisé
