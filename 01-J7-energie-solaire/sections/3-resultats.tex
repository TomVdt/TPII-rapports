\section{Résultats}

Il y a une relation linéaire entre \frac{1}{d²} et la puissance lumineuse par surface \(P\gamma\) sur le capteur. Un fit linéaire de la bibliothèque python numpy a été utilisé afin de mettre en évidence cette relation.
(  TODO:::::  La relation entre ces valeurs est Pgamma = alpha/d² avec alpha = (truc +/- machin)W.s^(-1)   )





\begin{itemize}
\item Présentation des résultats des mesures avec conditions dans lesquelles les mesures ont été effectuées, calculs des grandeurs dérivées et des incertitudes.
\item Tableaux et graphiques avec légendes incorporés au texte, unités physiques, rectangles d'erreurs.
\item La section résultats est purement descriptive: elle liste clairement les résultats obtenus : graphiques/tableaux avec les paramètres utilisés pour leur obtention (un externe doit pouvoir reproduire toutes vos mesures avec votre rapport).
\item Si certains points ne sont pas pris en compte, pour une raison justifiée, cela doit aussi être précisé.
\end{itemize}