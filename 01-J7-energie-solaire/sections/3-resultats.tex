\section{Résultats}

\paragraph*{Relation entre \(P_\gamma\) et \(\frac{1}{d^2}\)}
La puissance totale sortant de la lampe \(P_{tot}\) est constante. Cette puissance est répartie également sur toute la surface de la sphère correspondante à chaque instant de la propagation. Ainsi
\[P_{surf} = \frac{P_{tot}}{4 \pi r^2} = \frac{P_{tot}}{4 \pi} \frac{1}{d^2} \]
donnant alors une relation linéaire entre \(P_\gamma = P_{surf} S\) et \(\frac{1}{d^2}\), où \(P_{surf}\) est la puissance recue par unité de surface.
Cette relation linéaire, pour des grandes valeurs de d, est mise en évidence dans la figure \ref{plot:1}.

\begin{figure}[h]
    \centering
    \includegraphics[width=0.45\textwidth]{plots/Pgamma(1d2).tikz}
    \caption{Relation inversement quadratique entre distance et puissance lumineuse}
    \label{plot:1}
\end{figure}

\paragraph*{Courbes caractéristiques \(I_r(U_D)\) et \(P(U_D)\)}
La mesure de l'intensité du courant inverse de la diode \(I_R\) en fonction de la tension \(U_D\) aux bornes d'une résistance \(R_C\) branchée aux bornes de la diode permet de trouver la puissance de la cellule. Le courant \(I_R\) en fonction de \(U_D\) pour les différentes technologies de cellule est représenté à la Fig. \ref{plot:2}, pour deux conditions d'éclairement différentes.

\begin{figure}[h]
    \centering
    \begin{subfigure}[c]{0.4\linewidth}
        \centering
        \includegraphics[width=\textwidth]{plots/Ir(U),A.tikz}
        \caption{Amorphe}
        \label{plot:2a}
    \end{subfigure}
    \begin{subfigure}[c]{0.4\linewidth}
        \centering
        \includegraphics[width=\textwidth]{plots/Ir(U),MP.tikz}
        \caption{Mono- et polycristallin}
        \label{plot:2b}
    \end{subfigure}
    \caption{Caractéristique des cellules photovoltaïques}
    \label{plot:2}
\end{figure}

\paragraph*{Puissance et rendement max}
La puissance \(P\) fournie par la cellule est donnée par \(P = I_R U_D\). La valeur de \(P_{max}\) est obtenue en considérant le point extremum de la fonction \(P(U_D)\), illustré à la Fig. \ref{plot:3}. Les valeurs de \(P_{max}\) sous les différentes conditions sont données à la Fig. \ref{tab:pmax}

\begin{figure}[h]
    \centering
    \begin{subfigure}[c]{0.45\linewidth}
        \centering
        \includegraphics[width=\linewidth]{plots/P(U),A.tikz}
        \caption{Amorphe}
        \label{plot:3a}        
    \end{subfigure}
    \begin{subfigure}[c]{0.45\linewidth}
        \centering
        \includegraphics[width=\linewidth]{plots/P(U),MP.tikz}
        \caption{Mono- et polycristallin}
        \label{plot:3b}        
    \end{subfigure}
    \caption{Puissance \(P\) fournie par les cellules photovoltaïques}
    \label{plot:3}
\end{figure}

\begin{figure}[h]
    \centering
    \begin{tabulary}{0.7\linewidth}{C C C}
        \toprule
        Cellule & \(d_1\) & \(d_2\) \\
        \midrule
        Amorphe & \((0.075 \pm 0.074)\) \unit{\watt} & \((0.029 \pm 0.071)\) \unit{\watt} \\
        Monocristalline & \((0.069 \pm 0.007)\) \unit{\watt} & \((0.028 \pm 0.007)\) \unit{\watt} \\
        Polycristalline & \((0.033 \pm 0.005)\) \unit{\watt} & \((0.013 \pm 0.003)\) \unit{\watt} \\
        \bottomrule
    \end{tabulary}
    \caption{Valeurs de \(P_{max}\)}
    \label{tab:pmax}
\end{figure}

Les rendements maximaux sont donnés par \(\eta_{max} = \frac{P_{max}}{P_\gamma}\), avec \(P_\gamma = (1490 \pm 60)\) \unit{\milli\watt} pour \(d_1\) et \(P_\gamma = (621 \pm 33)\) \unit{\milli\watt} pour \(d_2\). La valeurs des rendements max sont donnés à la Fig. \ref{tab:rendementmax}. Le rendement max de la cellule monocristalline et amorphe sont similaires, alors que le rendement de la cellule polycristalline est deux fois plus faible que les autres.

\begin{figure}[h]
    \centering
    \begin{tabulary}{0.7\linewidth}{C C C}
        \toprule
        Cellule & \(d_1\) & \(d_2\) \\
        \midrule
        Amorphe & \((5.0 \pm 0.3)\)\% & \((4.6 \pm 0.3)\)\% \\
        Monocristalline & \((4.6 \pm 0.7)\)\% & \((4.5 \pm 1.0)\)\% \\
        Polycristalline & \((2.2 \pm 0.4)\)\% & \((2.0 \pm 0.6)\)\% \\
        \bottomrule
    \end{tabulary}
    \caption{Valeurs de \(\eta_{max}\)}
    \label{tab:rendementmax}
\end{figure}

Les rendements maximums pour les même cellules sont similaires. De plus la cellule monocristalline a un rendement maximum deux fois supérieur à celle polycristalline.

\paragraph*{Grandeur caractéristique moyenne \(i = I_\gamma/P_\gamma\)}
Le courant court-circuit \(I_\gamma\) est mesuré en enlevant la resistance variable \(R_C\) du circuit. Le courant \(I_\gamma\) varie fortement selon la distance \(d\), les distances courtes ayant une influence beaucoup plus grande que les distances plus longues, comme illustré à la Fig. \ref{plot:4}. \\
Une relation linéaire entre \(I_\gamma\) et \(P_\gamma\) peut être observée, et permet de trouver une grandeur caractéristique moyenne \(i = \frac{I_\gamma}{P_\gamma}\) pour chaque cellule photovoltaïque. La pente de la regression linéaire donne la valeur de \(i\) pour la cellule, les valeurs sont données dans la Fig. \ref{plot:5}

\begin{figure}[h]
    \centering
    \includegraphics[width=0.4\textwidth]{plots/Igamma(d),AMP.tikz}
    \caption{Courant photovoltaïque pour différentes cellules}
    \label{plot:4}
\end{figure}

\begin{figure}[h]
    \centering
    \begin{subfigure}[t]{0.48\linewidth}
        \centering
        \includegraphics[width=\textwidth]{plots/i(Pgamma),A.tikz}
        \caption{Amorphe}
        \label{plot:5a}
    \end{subfigure}
    \begin{subfigure}[t]{0.48\linewidth}
        \centering
        \includegraphics[width=\textwidth]{plots/i(Pgamma),MP.tikz}
        \caption{Mono- et polycristallin}
        \label{plot:5b}
    \end{subfigure}
    \caption{Grandeur caractéristique i pour différentes cellules}
    \label{plot:5}
\end{figure}

\paragraph*{Role du spectre lumineux}

Une autre caractéristique des cellules photovoltaïques à analyser est leur sensibilité à la composition de la lumière. La première mesure qui a été effectuée est donc celle de la variation de la puissance reçue en fonction du filtre placé devant la lampe à mercure, à une distance arbitraire de \(d_3 = (42.0 \pm 0.2)\) \unit{\centi\meter}. Cela a été fait pour les 7 filtres du tableau en annexe \ref{tab:filters}. Les mesures du courant de court-circuit \(I_\gamma\) pour chaque filtre et chaque cellule permettent de déterminer la valeur de \(i\) pour chacune de ces compositions de la lumière comme présenté dans les Fig. \ref{plot:6a} et \ref{plot:6b}. La plus basse valeur de \(i\) est donnée par le filtre 87C ne laissant passer que la lumière infrarouge. Le filtre 2B ne bloquant que les ultraviolets ne fait que peu baisser la valeur de \(i\).

\begin{figure}[h]
    \centering
    \begin{subfigure}[t]{0.45\linewidth}
        \centering
        \includegraphics[width=\textwidth]{plots/i(filtre),A.tikz}
        \caption{Amorphe}
        \label{plot:6a}
    \end{subfigure}
    \begin{subfigure}[t]{0.45\linewidth}
        \centering
        \includegraphics[width=\textwidth]{plots/i(filtre),MP.tikz}
        \caption{Mono- et polycristallin}
        \label{plot:6b}
    \end{subfigure}
    \caption{Grandeur caractéristique i des cellules photovoltaïques pour différents spectres}
    \label{plot:6}
\end{figure}

\paragraph*{Role de l'angle d'incidence}
Finalement la mesure de l'influence de l'angle d'incidence sur le courant fourni \(I_\gamma\) a été effectué en faisant varier l'angle du panneau par rapport à l'axe avec la lampe de \(0^{\circ}\) à \(90^{\circ}\). Les mesures sont rapportées dans la Fig \ref{plot:7} et un fit sinusoïdal a été appliqué. Il n'est pas nécessaire de faire les mesures pour des angles négatifs (faire tourner la cellule photovoltaïque dans l'autre sens) car la sinusoïde est symétrique.

\begin{wrapfigure}{R}{10cm}
    \includegraphics[width=\linewidth, height=6cm]{plots/Igamma(theta),MP.tikz}
    \caption{Intensité de sortie des cellules photovoltaïques en fonction de l'angle d'incidence.}
    \label{plot:7}
\end{wrapfigure}




% Présentation des résultats des mesures avec conditions dans lesquelles les mesures ont
%  été effectuées, calculs des grandeurs dérivées et des incertitudes.
% — Tableaux et graphiques avec légendes incorporés au texte, unités physiques, rectangles
% d’erreurs.
% — La section résultats est purement descriptive : elle liste clairement les résultats obtenus :
% graphiques/tableaux avec les paramètres utilisés pour leur obtention (un externe doit
% pouvoir reproduire toutes vos mesures avec votre rapport).
% — Si certains points ne sont pas pris en compte, pour une raison justifiée, cela doit aussi
% être précisé
