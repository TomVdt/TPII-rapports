\section{Théorie}

Les jonctions PN sont la base des capteurs photovoltaïques. Elles sont composées d'une double couche de semi-conducteurs dopés, qui, quand un photon est absorbé crée un "trou" positif.
Dans les cellules photovoltaïques, ces électrons libérés, le "trou" positif, sont collectés et s'accumulent afin de créer un courant et d'alimenter les appareils connectés au circuit.
Le courant \(I_D\) traversant la jonction PN est donné par l'equation \ref{eq:1} \cite{notice}.

\begin{equation}
    I_D = I_{SO}[e^{\alpha\frac{U_D}{T}} - 1] - I_\gamma
    \label{eq:1}
\end{equation}

où \(U_D\) est la tension directe, \(\alpha\) est une constante, \(T\) la température de la jonction en kelvins, \(I_{SO}\) le courant de saturation inverse de la jonction et \(I_\gamma\) le courant photovoltaïque.

ChatGPT dit:
Lorsque la lumière du soleil frappe la cellule photovoltaïque, elle génère des paires électron-trou dans la région de la jonction PN. Les électrons excitées dans la zone N sont poussés vers la jonction par le champ électrique créé par les charges fixes dans le matériau. Ces électrons se déplacent vers la zone P, créant ainsi un courant électrique.

Le courant électrique est collecté à partir de la cellule par des contacts métalliques connectés aux régions P et N. Lorsque les électrons circulent à travers une charge externe, ils font un travail contre la résistance, produisant ainsi de l'électricité.

blabla quoicoubeh est un trou uwu

Le rendement énergétique \(\eta\) d'une cellule solaire est donnée par \ref{eq:2}.

\begin{equation}
    \eta = \frac{P}{P_\gamma}
    \label{eq:2}
\end{equation}

avec \(P\) la puissance de sortie de la cellule et \(P_\gamma\) la puissance solaire reçue sur la surface de la cellule.

Expliquer

Fonctionnement des cellules

Jonctions PN

