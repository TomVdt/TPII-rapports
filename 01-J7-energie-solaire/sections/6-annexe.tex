\section{Caractéristiques des filtres}

\begin{figure}[H]
\centering
\begin{tabulary}{0.8\linewidth}{C C C}
    \toprule
    Filtre & Longueurs d'onde dans la bande passante & Transmissivité dans la bande passante \\
    \midrule
    2B  & $\lambda > 400 \unit{\nano \metre}$ & $0.9$ \\
    8   & $\lambda > 500 \unit{\nano \metre}$ & $0.9$ \\
    16  & $\lambda > 550 \unit{\nano \metre}$ & $0.9$ \\
    25  & $\lambda > 600 \unit{\nano \metre}$ & $0.9$ \\
    87C & $\lambda > 850 \unit{\nano \metre}$ & $0.9$ \\
    47  & $500 \unit{\nano \metre} > \lambda > 400 \unit{\nano \metre}$ & $0.9$ \\
        & $\lambda > 700 \unit{\nano \metre}$ & $0.5$ \\
    58  & $600 \unit{\nano \metre} > \lambda > 500 \unit{\nano \metre}$ & $0.9$ \\
        & $\lambda > 700 \unit{\nano \metre}$ & $0.5$ \\
    \bottomrule
    \end{tabulary}
    \caption{Caractéristiques des filtres \cite{notice}}
    \label{tab:filters}
\end{figure}

\section{Calcul d'erreurs}

Les erreurs estimées de mesure sont indiquées dans la figure \ref{tab:erreurs}

\begin{figure}[H]
    \centering
    \begin{tabulary}{0.7\linewidth}{C C C C C C C}
        \toprule
        Variable & \(d\)      & \(P_{\gamma,S}\) & \(U_D\)   & \(I_R\)  & \(l_{cellule}\) & \(\theta\) \\
        Erreur   & \(0.2\) cm & \(1\) \unit{\watt \per \square \meter}        & \(5\) \unit{\milli\volt} & \(3\) \unit{\milli\ampere} & 0.2 \unit{\centi\meter}      & \(1 \deg\) \\
        \bottomrule
    \end{tabulary}
    \caption{Erreurs estimées sur les mesures}
    \label{tab:erreurs}
\end{figure}

La formule utilisée pour le calcul d'erreur d'une fonction \(G(x,y,z)\) est \cite{erreursmesure}

\begin{equation*}
    \Delta G = \left|\frac{\partial{G}}{\partial{x}}\Delta x\right| + \left|\frac{\partial{G}}{\partial{y}}\Delta y\right| + \left|\frac{\partial{G}}{\partial{z}}\Delta z\right|
\end{equation*}

\paragraph*{Puissance \(P\)}
\[ \Delta P = \Delta (U_D I_R) = I_R \Delta U_D + U_D \Delta I_R \]

\paragraph*{Distance \(\frac{1}{d^2}\)}
\[ \Delta \frac{1}{d^2} = \frac{2}{d^3}\Delta d\]

\paragraph*{Surface \(S\)}
\[ \Delta S = 2 l_{cellule} \Delta l_{cellule} \]

\paragraph*{Grandeur caractéristique \(i\)}
\[ \Delta i = \Delta \frac{I_\gamma}{P_\gamma} = \frac{\Delta I_\gamma}{P_\gamma} + \frac{I_\gamma}{P_\gamma^2} \Delta P_\gamma \]

\paragraph*{Efficacité \(\eta\)}
\[ \Delta \eta = \Delta \frac{P}{P_\gamma} = \frac{\Delta P}{P_\gamma} + \frac{P}{P_\gamma^2} \Delta P_\gamma\]

\paragraph*{Régression linéaire}
Pour trouver une droite \(y = ax+b\) par la méthode des moindres carrés,
il faut appliquer la formule:

\[ a=\frac{\sum_{i}(x_{i}-\bar{x})(y_{i}-\bar{y})}{\sum_{i}(x_{i}-\bar{x})^{2}} \]

\[ b = \bar{y} - a\bar{x}\]

L'erreur sur la pente nous est donnée par

\[ \left(\Delta{a}\right)^{2}\,=\,{\frac{\sum_{i}(y_{i}-\,a{x}_{i}-\,b)^{2}}{\left(n-\,2\right)\cdot\,\sum_{i}({x}_{i}\,-\,\bar{C})^{2}}} \]

avec n le nombre de points. En pratique, les erreurs sur les régressions ont été calculées par la bibliothèque \texttt{numpy}.
