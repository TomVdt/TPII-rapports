\section{Caractéristiques des filtres}

% \begin{figure}[h]
% \centering
% \begin{tabulary}{0.7\linewidth}{C C C}
%     \toprule
%     Filtre & Longueurs d'onde dans la bande passante & Transmissivité dans la bande passante \\
%     \midrule
%     2B  & $\lambda > 400 \unit{\nano \metre}$ & $0.9$ \\
%     8   & $\lambda > 500 \unit{\nano \metre}$ & $0.9$ \\
%     16  & $\lambda > 550 \unit{\nano \metre}$ & $0.9$ \\
%     25  & $\lambda > 600 \unit{\nano \metre}$ & $0.9$ \\
%     87C & $\lambda > 850 \unit{\nano \metre}$ & $0.9$ \\
%     47  & $500 \unit{\nano \metre} > \lambda > 400 \unit{\nano \metre}$ & $0.9$ \\
%         & $\lambda > 700 \unit{\nano \metre}$ & $0.5$ \\
%     58  & $600 \unit{\nano \metre} > \lambda > 500 \unit{\nano \metre}$ & $0.9$ \\
%         & $\lambda > 700 \unit{\nano \metre}$ & $0.5$ \\
%     \bottomrule
%     \end{tabulary}
%     \caption{Caractéristiques des filtres \cite{notice}}
%     \label{tab:filters}
% \end{figure}

\section{Calcul d'erreurs}


Incertitude sur \(\frac{1}{d^2}\): 
\begin{equation*} 
    \Delta \frac{1}{d^2} = \left|\frac{\partial\frac{1}{d^2}}{\partial d}\Delta d\right| = \left|\frac{-2\Delta d}{d^3}\right| = \frac{2\Delta d}{d^3} 
\end{equation*}
Incertitude sur Pgamma donnée par la notice de l'instrument (le laboratoire est estimé être à 25°C): le maximum de +/-10 W/m² et +/- 5%
(   TODO::::::    De plus nous obtenons grâce à la fonction polyfit() l'incertitude sur cette valeur à partir des incertitudes sur 1/d² et Pgamma)



\begin{itemize}
\item L'annexe n'est utile que pour le matériel supplémentaire, notamment pour les tableaux des paramètres utilisés ou encore les calculs d'erreur (seulement dans les rapports, pas dans un article scientifique).
\item Toutes les figures d'intérêt doivent figurer dans le corps du rapport.
\item Tous les éléments figurant dans l'annexe doivent être cités à l'endroit opportun dans les sections précédentes.
\end{itemize}

% \begin{itemize}
    % \item \Delta U
    % \item \Delta I
    % \item 
% \end{itemize}