\section{Conclusion}

Au long de cette expérience, plusieurs caractéristiques des cellules photovoltaïque on été mises en avant.
Dans un premier temps, une relation entre puissance lumineuse et \(\frac{1}{d^2}\) a été établie. La mesure de la tension \(U_D\) et l'intensité \(I_R\) pour différentes résistances ont permis de trouver la puissance maximale, et ainsi le rendement maximal des trois cellules. Leur sensibilité aux différentes portions du spectre lumineux a permis de connaitre le spectre idéal pour ces cellules, correspondant à une lumière jaune-orange.
Si la cellule amorphe produit un courant d'intensité bien plus faible que les cellules cristallines, elle opère cependant bien mieux sous une grosse charge (\(U_{max}\) est grand), alors que les mono- et polycristallines génèrent un fort courant sous de petites charges (\(U_{max}\) est petit).
L'angle d'incidence affecte peu l'intensité de sortie de toutes les cellules, pour des petits angles.
L'étude des différentes technologies de cellules photovoltaïque favorise le developpement de cellules de plus en plus efficaces, en utilisant par exemple une plus grande portion du spectre visible, et pourrait être un enjeu clé dans le developpement d'un réseau électrique stable et respectueux de l'environment.


% \begin{itemize}
% \item Résumé, dense, du travail. Réponse claire à l'introduction.
% \item \textit{Objectifs atteints}?
% \item La conclusion doit remettre vos résultats dans le contexte mentionné dans l'introduction et montrer en quoi vos mesures sont importantes/utiles/intéressantes et applicables (dans le projet). N'hésitez pas à remettre les valeurs numériques d'intérêt dans le texte pour appuyer vos déductions.
% \item Il ne faut pas commencer par "En conclusion, … ", c'est évident du titre de la section
% \end{itemize}