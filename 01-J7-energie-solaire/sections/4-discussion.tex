\section{Discussion}



\paragraph{}
Divers rendements maximums pour les trois technologies de cellule ont été obtenus. Pour chaque type de cellule la différence entre les rendements à des distances différentes était de l'ordre de \(10^{-3}\) ce qui est preuve d'une bonne cohérence des résultats obtenus. De plus, il a été trouvé que les rendements des cellules amorphes et monocristallines sont extrêmement proches, tandis que la cellule polycristalline possède un rendement maximum deux fois plus faible.

\paragraph{}
Les graphiques \ref{plot:5a} et \ref{plot:5b} obtenus de \(I_\gamma (P_\gamma)\) donnent des relations linéaires entre ces valeurs. NUMERO 4




Principalement sensible à la lumière visible. NUMERO 5


INCIDENCE  BONUS 1


\paragraph{Données du fabricant}
Le fabricant de la cellule monocristalline utilisée dans cette expérience indique un rendement de 10\% sous éclairage de \(1000\) \unit{\watt \per \square \meter}. Bien que les valeurs du rendement établies lors de cette expérience sont bien plus faibles que celles indiquées, de 4.5\% à 4.7\%, les conditions d'éclairage ne sont pas les mêmes. Déjà, l'éclairage maximal atteint dans le laboratoire était \((283 \pm 1)\) \unit{\watt \per \square \meter}, bien plus faible que les \(1000\) \unit{\watt \per \square \meter} indiqués. Ce rendement semble plausible, étant donné qu'avec moins d'un tiers de l'éclairage, le rendement était la moitié de la valeur indiquée.
