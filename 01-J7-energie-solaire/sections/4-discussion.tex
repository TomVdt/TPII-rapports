\section{Discussion}

La puissance totale qui sort de la lampe \(P_{tot}\) est constante. Cette puissance est répartie également sur toute la surface de la sphère correspondante à chaque instant de la propagation. Ainsi \(P_{surf} = \frac{P_{tot}}{4 \pi r^2} = \frac{P_{tot}}{4 \pi} \frac{1}{d^2} \). Ce qui donne logiquement une relation linéaire entre \(P_\gamma = P_{surf} S\) et \(\frac{1}{d^2}\).

\paragraph{}
Divers rendements maximums pour les trois technologies de cellule ont été obtenus. Pour chaque type de cellule la différence entre les rendements à des distances différentes était de l'ordre de \(10^{-3}\) ce qui est preuve d'une bonne cohérence des résultats obtenus. De plus, il a été trouvé que les rendements des cellules amorphes et monocristallines sont extrêmement proches, tandis que la cellule polycristalline possède un rendement maximum deux fois plus faible.

\paragraph{}
Les graphiques obtenus de \(I_\gamma (P_\gamma)\) donnent des relations linéaires entre ces valeurs. NUMERO 4




Principalement sensible à la lumière visible. NUMERO 5


INCIDENCE  BONUS 1


\paragraph{}
En ce qui concerne le rapport avec les valeurs du fabricant, la cellule monocristalline possède un rendement allant de 4.5\% à 4.7\% comme mesuré en laboratoire et cette valeur n'est que peu modifiée par des variations puissance lumineuse. Il semble peu probable que la légère augmentation observée permette d'atteindre un rendement de 10\% à 1000 \unit{\watt\per\meter^2}. La différence de rendement peut s'expliquer par l'imprécision en laboratoire, les erreurs humaines ou simplement un montage pas suffisament précis. \\
En effet le montagé réalisé comporte de nombreux éléments pouvant perturber la mesure, tel que les résistances internes des appareils de mesure ou l'imprécision de certains outils utilisés. Par exemple l'utilisation d'un fil métallique pour obtenir une certaine résistance peut s'avérer très imprécis car pas indiqué avec suffisament de précision. Ces éléments auraient plutôt tendance à diminuer les valeurs mesurées. \\
Un autre élément pouvant perturber les valeurs positivement ou rajoutant simplement du bruit aux mesures est la présence de murs blancs et d'une fenêtre non-bloquée à proximité de l'appareil. Bien que ces éléments soient souvent négligeable grâce à la grande puissance de la lampe à mercure il se peut que dans les mesures plus sensibles tel que celle de l'intensité en fonction de l'angle d'incidence cela ait eu une influence.
