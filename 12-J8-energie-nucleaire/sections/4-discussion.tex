\section{Discussion}

\paragraph{Approche critique}
TODO mais probs linéaireish, avec un commentaire sur le faible nombre de mesures perhaps

\paragraph{Source de neutrons}
Il a été observé expérimentalement que le réacteur atteint la criticalité pour une hauteur du modérateur (l'eau) de \(h_c = 952.2 \pm 0.1\) mm. Dans cette état, le coeur du réacteur est capable de maintenir la réaction en chaine, sans causer d'augmentation exponentielle du nombre de réactions. La source de neutrons qui avait été utilisée pour lancer la réaction en chaine avant la criticalité n'est donc plus nécessaire. Cela a aussi été observé: à la hauteur critique, avec la source de neutrons présente, le flux neutronique augmentait linéairement. Une fois la source retirée, le flux neutronique s'est stabilisé et est devenu presque constant. Il y avait une légère tendance vers le bas, ce qui peut indique que, bien que très proche de l'état critique, cet état est dans les fait très compliqué a maintenir exactement. Il est aussi préférable de rester légèrement en dessous de la hauteur critique, pour éviter une soudaine augmentation exponentielle des réactions.

\paragraph{Hauteur et puissance}
feur