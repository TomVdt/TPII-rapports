\section{Discussion}

\paragraph{Approche critique}
L'approche de la hauteur critique a bien été effectuée avec le comportement attendu. La linéarité de $1/I$ et $1/C$ avec $h$ a bien été confirmée comme le montre les \autoref{fig:hc_intensite} et \autoref{fig:hc_compte}. Le détecteur d'intensité avait une erreur relative de l'ordre de grandeur de 10\% ce qui a donné en \autoref{fig:hc_I} de grandes erreurs mais qui comprenaient donc toujours la hauteur critique dans leur intervalle de confiance. La mesure avec le taux de comptage a également utilisée des erreurs de l'ordre de 10\% sur le compte $N$ ce qui a également permis de conserver la valeur réelle dans les intervalles de confiance. Un point d'amélioration, nécessaire probablement dans une centrale de plus grande importance pouvant présenter de réels risques, serait d'effectuer plus de mesures afin de pouvoir afiner la détermination de la hauteur critique en s'en rapprochant toujours plus. Il reste cependant important de toujours prendre des hauteurs significativement inférieures à celle estimée pour tester car comme illustré dans les \autoref{fig:hc_I} et \autoref{fig:hc_C} il est fréquent que $h_C$ soit sur-estimé.

\paragraph{Source de neutrons}
Il a été observé expérimentalement que le réacteur atteint la criticité pour une hauteur du modérateur (l'eau) de \(h_c = 952.2 \pm 0.1\) mm. Dans cette état, le coeur du réacteur est capable de maintenir la réaction en chaine, sans causer d'augmentation exponentielle du nombre de réactions. La source de neutrons qui avait été utilisée pour lancer la réaction en chaine avant la criticité n'est donc plus nécessaire. Cela a aussi été observé: à la hauteur critique, avec la source de neutrons présente, le flux neutronique augmentait linéairement. Une fois la source retirée, le flux neutronique s'est stabilisé et est devenu presque constant. Il y avait une légère tendance vers le bas, ce qui peut indiquer que, bien que très proche de l'état critique, cet état est dans les fait très compliqué à maintenir exactement. Il est aussi préférable de rester légèrement en dessous de la hauteur critique, pour éviter une soudaine augmentation exponentielle des réactions.

\paragraph{Hauteur et puissance}
Une dernière observation a pu être effectuée sur le comportement du réacteur dans les états sur- et sous-critiques. En effet, une fois la hauteur $h_C$ trouvée il a été possible de la dépasser afin d'observer l'augmentation exponentielle de la puissance du réacteur et du flux de neutrons $\Phi$. En quelques minutes une puissance de 1W a pu être atteinte alors que toutes les mesures ont été faites avec des puissances de l'ordre du \si{\milli\watt}. Il a ensuite de manière similaire été possible de diminuer sa puissance en le mettant dans un état sous-critique avec un niveau d'eau inférieur à $h_C$. Il est donc tout a fait possible de controler la puissance d'un réacteur, pour des applications énergétiques par exemple, à l'aide des état sous- et sur-critiques en modifiant son facteur de multiplication $k_\mathrm{eff}$ par le biais de barres de controle ou du niveau du modérateur.