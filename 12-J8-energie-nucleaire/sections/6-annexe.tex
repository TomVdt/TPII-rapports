\section{Calcul d'erreurs}
\label{sec:erreurs}

Les erreurs sur les mesures sont données dans le \autoref{tab:erreurs}.

\begin{table}[h]
    \centering
    \begin{tabulary}{\textwidth}{C C}
        \toprule
        Variable & Erreur \\
        \midrule
        Temps \(t\) [s] & 0.1 \\
        Hauteur \(h\) [mm] & 0.1 \\
        \bottomrule
    \end{tabulary}
    \caption{Erreurs estimées sur les mesures}
    \label{tab:erreurs}
\end{table}

\paragraph*{Regression linéaire}
Les erreurs sur les régressions linéaires \(y = ax + b\) sur les mesures \((x_i, y_i) ; i = \{1, \dots, n\}\) sont donnés par \cite{erreursmesure}:

\begin{equation}
    \label{eq:erreur:fit}
    \begin{aligned}
        (\Delta a)^2 &= \frac{\sum_{i=1}^{n}(y_i - (a x_i + b))^2}{(n-2) \sum_{i=1}^{n}(x_i - \bar{x})^2}\\
        \Delta b &= \bar{x} \Delta a + a \Delta \bar{x}
    \end{aligned}
\end{equation}

En pratique, ces valeurs sont calculées par la bibliothèque python \texttt{numpy}.

\paragraph*{Formules d'erreurs}

Erreur sur le nombre de neutrons \(n\). En estimant que les neutrons sont libérés selon une loi de Poisson, \(n \sim \textrm{Pois}(\lambda)\), on a \(\bar n = \textrm{Var}(n)\) et alors:
\begin{equation}
    \Delta n = n \frac{1}{\sqrt{n}} = \sqrt{n}
\end{equation}

Erreur sur le taux de compte \(C\):
\begin{equation}
    \Delta C = \absfrac{\Delta n}{n} C + \absfrac{\Delta t}{t} C
\end{equation}

Ces erreurs sont calculées en pratique par la bibliothèque \texttt{uncertainties}.

L'estimation de l'erreur sur \(h_c\) est réalisée graphiquement, à partir de paires de points. L'estimation basse est donnée on considérant la borne supérieure du premier point (\(h_i, 1/C_i + 1/\Delta C_i\)) et l'estimaton basse du second (\(h_{i+1}, 1/C_{i+1} - 1/\Delta C_{i+1}\)), et inversement pour l'estimation haute. En réalisant une régression linéaire, les valeurs \(h_{c,\textrm{min}}\) et \(h_{c,\textrm{max}}\) sont données par l'abscisse de l'intersection avec l'axe x.

