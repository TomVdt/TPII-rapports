\begin{wrapfigure}{R}{0.42\linewidth}
    \includegraphics*[width=\linewidth]{figures/crocus.jpg}
    \caption{Le réacteur CROCUS à l'arrêt le 23 Mai 2024}
    \label{fig:crocus}
\end{wrapfigure}
\section{Démarche Expérimentale}
Le réacteur nucléaire CROCUS en \autoref{fig:crocus} est un réacteur expérimental, dont la puissance maximale autorisée est 100W. Afin d'atteindre la criticité, le niveau d'eau dans la cuve du réacteur peut être controlée par un.e opérateur.trice. Un niveau d'eau initial \(h_0 = 920.0 \pm 0.1\) mm est choisi, puis après stabilisation du compte de neutrons, l'intensité $I$ du courant est relevée et le temps \(t\) nécessaire jusqu'à un compte \(N\) de 10 000 neutrons détectés est aussi mesuré. Ce compte a été choisi afin d'avoir une erreur relative sur \(N\) inférieure à 1\%. Les détails sur ce calcul d'erreur sont disponibles en \autoref{sec:erreurs}. Le taux de comptage \(C\) peut alors être calculé. Ensuite, afin d'estimer la hauteur critique \(h_c\), une régression linéaire sur deux points \((h, \frac{1}{C})\) est réalisée. L'intersection entre cette régression et l'axe x, c'est a dire pour un taux de comptage tendant vers l'infini, donne alors une estimation de la hauteur critique \(h_c\). Le même procédé est réalisé avec l'inverse du courant $1/I$. Afin de ne pas dépasser la hauteur critique, la hauteur choisie pour la prochaine mesure est telle que:
\begin{equation}
    h_{i+1} = h_i + \frac{1}{3}(h_{c,i} - h_i)
\end{equation}
où \(h_{c,i}\) est la hauteur critique déterminée à partir des mesures aux hauteurs \(h_{i-1}\) et \(h_{i}\), \(i=1,2,3\). Les mesures entre chaque changement de niveau doivent être réalisées en prenant soin que le compte se soit stabilisé pour prendre en compte les neutrons retardés.

