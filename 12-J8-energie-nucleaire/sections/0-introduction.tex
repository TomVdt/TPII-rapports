\section{Introduction}

En 2023, la consommation mondiale d'électricité a atteint près de 30 000 TWh \cite{owid-energy}. La manière dont cette électricité est produite représente un enjeu politique et climatique majeur. Actuellement, 61\% de la production d'électricité provient de sources fossiles, causant une pollution significative et l'épuisement des ressources naturelles, tandis que seulement 10\% provient du nucléaire \cite{owid-electricity-mix}. Bien que l'énergie nucléaire présente des inconvénients, tels que la production de déchets radioactifs, elle reste l'option la moins émettrice de gaz à effet de serre par kWh produit \cite{owid-nuclear-energy}.
Face aux défis énergétiques et climatiques, la transition vers des sources d'énergie plus propres est cruciale. Les énergies renouvelables sont intermittentes et nécessitent des solutions de stockage et une modernisation des infrastructures. Le nucléaire, malgré ses défis, offre quant à lui une production continue et à faible émission de carbone. Il est donc essentiel de maitriser l'exploitation de cette source d'énergie.

Les résultats obtenus sur le réacteur à fission expérimental CROCUS seront présentés dans ce rapport, en plus de ses principes de fonctionnement. Le niveau d'eau amenant au seuil de criticité sera notament déterminé.