\section{Démarche Expérimentale}

Afin de mesurer les capacités plastiques d'un solide, plusieurs échantillons d'un alliage d'aluminium (Anticorodal 110) de dimensions similaires sont utilisés. Ces échantillons ont une partie fine au milieu sur laquelle les mesures vont se concentrer et une partie large sur les extrémités pour permettre une bonne prise. Un dispositif permet d'appliquer et mesurer une force de traction sur les extrémités de l'échantillon. Une mesure de déformation est aussi réalisée sur la partie fine de l'échantillon à l'aide d'un capteur utilisant un effet de levier. Les tensions de sortie sont converties en utilisant les valeurs de calibration. Les points (déformation, force) obtenus sont alors affichées sur un graphique afin de permettre une analyse similaire à la \autoref{fig:courbe_theorie}.

\paragraph{Traitement} Afin d'observer les effets d'un traitement thermique sur les échantillons, trois états diférents sont mesurés. Dans un premier temps la mesure est réalisée sur des échantillons non traités. Ensuite, des échantillons sont placés à 550 \si{\celsius} pendant 1h afin de les adoucir. Ils sont ensuite refroidis rapidement dans un bac d'eau afin de conserver leur structure atomique. Des essais de tractions sont réalisés sur une partie de ces échantillons. Enfin, les échantillons restants sont remis à 250 \si{\celsius} pendant 20 min, pour les redurcir. Ils sont également refroidis rapidement.