\section{Introduction}
Les alliages de métaux ont eu une grande importance dans le développement de l'humanité avec les premiers pouvant remonter jusqu'a 3000 avant notre ère \cite{vieux}. De nos jours, leur usage ne faiblit pas, dans l'industrie aéronautique par exemple les alliages d'aluminium représentent 60 à 80\% de la masse d'un avion de ligne \cite{aluminium}. Il est donc important de pouvoir étudier ces matériaux afin de bien connaitre leurs caractéristiques pour une utilisation fiable. Un des points d'intérêts principaux est les déformations du matériau utilisé sous une contrainte. Ainsi les déformations élastiques et plastiques doivent être étudiées ce qui peut être fait en partie avec un essai de traction.

Dans cette expérience des essais de traction ont donc été réalisés sur plusieurs échantillons d'un alliage d'aluminium. Afin d'observer les effets d'un traitement thermique, certains de ces échantillons ont été chauffés de plusieurs façons et pour tous les déformations élastiques et plastiques ont été observées jusqu'à la rupture de l'échantillon. Cela a permis de déterminer le module de Young, la limite élastique, la résistance à la traction et la déformation à la rupture de ces alliages.


