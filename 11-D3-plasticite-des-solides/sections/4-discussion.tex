\section{Discussion}
\paragraph{Module de Young}
Les valeurs obtenues pour le module de Young sont assez proches des valeurs tabulées pour l'échantillon non-traité, avec un écart relatif de 18\%. Cette élasticité ne devrait pas changer suivant le traitement, ce qui est observé pour l'échantillon adouci où l'écart relatif n'est alors que de 14\%. Cependant, l'échantillon adouci puis durci diffère de plus de 50\% de la valeur tabulée. Cet écart peut s'expliquer par un problème observé lors de l'acquisition des données: initialement, la déformation de l'échantillon diminue lorsque la force augmente, alors qu'on s'attend à une pente positive. Cet effet est visible sur la \autoref{fig:tiede6_lim}. C'est la raison principale pour laquelle la plupart des mesures n'ont pas pu être utilisées et que seulement les résultats pour trois échantillons sont présentés. Cette expérience présente donc un défaut pouvant conduire à rendre inutilisables de nombreuses mesures.

\paragraph{Déformations rémanentes}
L'incertitude sur le module de Young s'est logiquement répercutée sur les relevés indirects des propriétés de la déformation rémanente. En effet celle-ci correspond à la déformation en descendant la ligne de pente $E$. Cependant les résultats finaux sont restés cohérents en ce qui concerne leurs ordres de grandeur aussi bien pour la limite élastique $\sigma_{0.2}$ que pour la déformation rémanente à la rupture $\varepsilon_\mathrm{rup}$ qui sont toutes deux des valeurs liées à la déformation rémanente et obtenues à partir du module de Young. Cela peut s'expliquer par le fait que dans tous les cas le module de Young obtenu était très élevé (de l'ordre de \mbox{$10^{10}$ \si{\pascal}}). Ainsi comme les courbes de traction obtenues avaient pour toutes les mesures la bonne allure globale, une pente très elevée comme attendue donne des valeurs cohérentes car elle s'oriente dans la bonne direction. Ainsi cette expérience bien qu'elle n'est pas montrée une grande fiabilité numérique pour certaines grandeurs confirme la proximité de la théorie avec l'expérience. Les courbes obtenues permettaient donc bien de distinguer les différents modes de déformation, élastique, plastique et striction. La ressemblance entre les courbes de la \autoref{fig:courbe_theorie} et de la \autoref{fig:tractions_exp} montrent bien cette cohérence entre théorie et expérience.

\paragraph{Rigidité des matériaux}
Le traitement thermique a de manière claire permis de modifier la ductilité des échantillons considérés. Un matériau plus ductile aura une limite élastique plus basse, car plus susceptible à la déformation plastique, cependant aura un point de rupture beaucoup plus éloigné car plus à même de supporter les déformations. Il sera également moins résistant à la contrainte ici la traction. Cela est en cohérence avec le fait que le matériau chauffé pour être adouci avait les plus basses limite élastique $\ligma_{0.2}$ et résistance à la traction $\ligma_\mathrm{max}$ mais la déformation à la rupture $\varepsilon_{\textrm{rup}}$ la plus élevée, deux fois plus grande que les autres échantillons, c'était donc le moins rigide. Le plus rigide toujours selon les même critères avec les valeurs du \autoref{tab:results} était l'échantillon non-traité puis entre les deux celui adouci puis redurci. Il est possible d'observer ces changements de comportements dans la \autoref{fig:comparaison} qui montre bien l'augmentation de ductilité avec la température du dernier traitement. Cela est en accord avec la théorie sur les dislocations et les défauts de la \autoref{sec:theorie} avec le matériau chauffé au maximum ayant les défauts les plus dissouds et donc la plus grande susceptibilité à la déformation plastique.

\paragraph{Effet Portevin - Le Chatelier}
Il est possible de voir sur la \autoref{fig:chaud3} des petits "sauts" sur la courbe de traction, qui ne sont pas présents pour les autres échantillons. Il s'agit de l'effet Portevin-Le Chatelier \cite{Yilmaz_2011}. A cause des interactions entre les atomes solutés et les dislocations, lorsqu'une force est appliquée, les dislocations se déplacent et se bloquent au niveau des atomes solutés. Lorsque la force dépasse un certain seuil, la dislocation dépasse le niveau de l'atome soluté et est à nouveau capable de se déplacer. La différence de force nécessaire pour le mouvement de la dislocation au passage d'un atome soluté est responsable des variations de force à appliquer, comme observé.