\section{Discussion}
\paragraph{Module de Young}


\paragraph{Déformations rémanentes}
La grande incertitude sur le module de Young s'est logiquement répercutée sur les relevés indirects des propriétés de la déformation rémanente. En effet celle-ci correspond à la déformation en descendant la ligne de pente $E$. Cependant les résultats finaux sont restés cohérents en ce qui concerne leurs ordres de grandeur aussi bien pour la limite élastique $\sigma_{0.2}$ que pour la déformation rémanente à la rupture $\varepsilon_\mathrm{rup}$ qui sont toutes deux des valeurs liées à la déformation rémanente et obtenues à partir du module de Young. Cela peut s'expliquer par le fait que même dans le cas où le module de Young obtenu était négatif finalement sa valeur restait très élevée (de l'ordre de $10^{10}$ \si{\pascal}). Ainsi comme les courbes de traction obtenues avaient pour toutes les mesures la bonne allure globale une pente très elevée, comme attendue, donne des valeurs cohérentes. Ainsi cette expérience bien qu'elle n'est pas montrée une grande fiabilité numérique pour certaines grandeurs confirme la proximité de la théorie avec l'expérience. Les courbes obtenues permettaient donc bien de distinguer les différents modes de déformation, élastique, plastique et striction.

\paragraph{Rigidité des matériaux}
Le traitement thermique a de manière claire permis de modifier la ductilité des échantillons considérés. Un matériau plus ductile aura une limite élastique plus basse, car plus susceptible à la déformation plastique, cependant aura un point de rupture beaucoup plus éloigné car plus à même de supporter les déformations. Il sera également moins résistant à la contrainte ici la traction. Cela est en cohérence avec le fait que le matériau chauffé pour être adouci avait les plus basses limite élastique $\ligma_{0.2}$ et résistance à la traction $\ligma_\mathrm{max}$ mais la déformation à la rupture $\varepsilon_r$ la plus élevée, presque deux fois plus grande que les autres échantillons, c'était donc le moins rigide. Le plus rigide toujours selon les même critères avec les valeurs du \autoref{tab:results} était l'échantillon non-traité puis entre les deux celui adouci puis redurci. Cela est en accord avec la théorie sur les dislocations et les défauts de la \autoref{sec:theorie} avec le matériau chauffé au maximum ayant les défauts les plus dissouds et donc la plus grande susceptibilité à la déformation plastique.


\paragraph{Effet Portevin - Le Chatelier}
TODO