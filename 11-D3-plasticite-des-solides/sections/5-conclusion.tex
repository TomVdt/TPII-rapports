\section{Conclusion}

Cette expérience a permis de déterminer les propriétés plastiques d'échantillons d'Anticorodal 110 soumis à différents traitements thermiques. A partir d'essais de traction, le module de Young \(E\) ainsi que la limite élastique \(\sigma_{0.2}\), la résistance à la traction \(\sigma_{\textrm{max}}\) et la déformation à la rupture \(\varepsilon_{\textrm{r}}\) ont été déterminés. Un comportement innatendu a été observé sur le module de Young. De plus, l'effet Portevin-Le Chatelier a été observé pour l'échantillon adouci. L'étude de ces traitements est essentielle dans l'aéronautique, où les propriétés élastiques et plastiques des matériaux sont déterministes à la structure des ailes d'avion par exemple \cite{aluminium}.