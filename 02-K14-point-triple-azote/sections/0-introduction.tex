\section{Introduction}


Les techniques du vide ont de nombreuses applications aussi bien dans la vie courante que dans l'industrie ou la recherche. \\
La mise sous vide des aliments permet leur conservation sur une longue durée notamment grâce à la réduction de l'oxygène disponible dans l'emballage empêchant la prolifération de bactéries ou fungi sur les aliments. Cette pratique est devenue si courantes que les dispositifs de mise sous vide personnels sont devenus bien répandus. \\
Dans le domaine de la recherche le vide est utilisé dans de très nombreux domaines car il permet de fournir un environnement inerte chimiquement, une excellente isolation thermique ou encore une grande distance sans collision pour une particule libre. Cela est notamment le cas de l'installation du CERN à Genève pour leur Grand collisionneur de hadrons qui utilise 104 kilomètres de conduits sous vide, dont 50 kilomètres pour l'isolation thermique de différents éléments s'additionant à 15 000 \unit{\cubic \meter} de vide à 10^{-6} \unit{\milli \bar} \cite{CERN}.

