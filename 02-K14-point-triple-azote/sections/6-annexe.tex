\section{Calcul d'erreurs}

Les erreurs sur les mesures sont données dans le \autoref{tab:erreurs}.

\begin{table}[h]
    \centering
    \begin{tabulary}{\textwidth}{C C C C}
        \toprule
        Variable & Valeur & Erreur & Commentaire \\
        \midrule
        \(V\) [\unit{\liter}] & 5.5 & 0.5 & Estimation basée sur la valeur indiquée \\
        \(p_\textrm{Bourdon}\) [\unit{\milli\bar}] & \(\emptyset\) & 10 & Moitié d'une graduation \\
        \(p_\textrm{Piranni}\) [\unit{\milli\bar}] & \(\emptyset\) & \(0.5 \times 10^\textrm{gamme}\) & Moitié d'une graduation pour une gamme \\
        \(p_\textrm{Penning}\) [\unit{\milli\bar}] & \(\emptyset\) & \(0.5 \times 10^\textrm{gamme}\) & Moitié d'une graduation pour une gamme \\
        \(p_\textrm{lim,1}\) [\unit{\milli\bar}] & \(1.0 \cdot 10^{-2}\) & \(0.5 \cdot 10^{-2}\) & Obtenu en considérant la valeur lue après 15min \\
        \(p_\textrm{lim,2}\) [\unit{\milli\bar}] & \(1.2 \cdot 10^{-5}\) & \(0.5 \cdot 10^{-5}\) & Obtenu en considérant la valeur lue après 15min \\
        \midrule
        \(T\) [\unit{\kelvin}] & \(\emptyset\) & 0.1 & Dernier chiffre significatif sur l'affichage \\
        \(P\) [\unit{\milli\bar}] & \(\emptyset\) & 1 & Dernier chiffre significatif sur l'affichage \\
        \(t\) [\unit{\second}] & \(\emptyset\) & 0 & Mesures prises en avancant \textit{frame by frame} \\
        \bottomrule
    \end{tabulary}
    \caption{Erreurs estimées sur les mesures}
    \label{tab:erreurs}
\end{table}

\paragraph*{Regression linéaire}
Les erreurs sur les fits linéaires \(y = ax + b\) sur les mesures \((x_i, y_i) ; i = \{1, \hdots, n\}\) sont donnés par l'\autoref{eq:erreur:fit} \cite{erreursmesure}:

\begin{equation}
    \label{eq:erreur:fit}
    \begin{aligned}
        (\Delta a)^2 &= \frac{\sum_{i=1}^{n}(y_i - (a x_i + b))^2}{(n-2) \sum_{i=1}^{n}(x_i - \bar{x})^2}\\
        \Delta b &= \bar{x} \Delta a + a \Delta \bar{x}
    \end{aligned}
\end{equation}

En pratique, ces valeurs sont calculées par la bilbiothèque \texttt{numpy}.

\paragraph*{Debit de pompage}
L'erreur sur le débit de pompage est donné par l'\autoref{eq:erreur:debit}:

\begin{equation}
    \Delta S = \frac{\Delta a}{V} + \frac{a}{V^2} \Delta V
    \label{eq:erreur:debit}
\end{equation}

où \(a\) et \(\Delta a\) viennent du fit linéaire de \(\ln{\frac{P - P_0}{P_1 - P_0}}\). L'erreur sur ces valeurs est donné par l'\autoref{eq:erreur:pression}:

\begin{equation}
    \Delta \ln{\frac{P - P_0}{P_1 - P_0}} = \left[\frac{\Delta P - \Delta P_0}{P_1 - P_0} + \frac{P - P_0}{(P_1 - P_0)^2}(\Delta P_1 + \Delta P_0)\right] \frac{P_1 - P_0}{P - P_0}
    \label{eq:erreur:pression}
\end{equation}

Toutes ces valeurs sont calculées en pratique par la bibliothèque \texttt{uncertainties}.
