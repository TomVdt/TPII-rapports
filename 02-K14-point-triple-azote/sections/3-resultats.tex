\section{Résultats}


\begin{minipage}{\textwidth}
    \begin{wrapfigure}{R}{0.5\textwidth}
        \centering
        \includegraphics[width=\linewidth]{figures/cinetique_palettes.pdf}
        \caption{Pression dans la chambre pour la pompe à palettes en fonction du temps}
        \label{fig:cinetique_palette}
    \end{wrapfigure}
    
    \paragraph*{Débit de la pompe à palettes}
    La pression dans la chambre de volume \(V = (5.5 \pm 0.5)\) \si{\milli\bar} en fonction du temps pour la pompe primaire est donné dans la \autoref{fig:cinetique_palette}. Une régression linéaire sur les points entre \(t = 6\) \si{\second} et \(t = 28\) \si{\second} permet d'obtenir une valeur moyenne sur \(\ln{\frac{P - P_0}{P_1 - P_0}}\), où \(P_1 = 620 \pm 10\) \si{\milli\bar}. La pente de cette droite permet d'obtenir avec l'\autoref{eq:cinetique} le débit effectif de la pompe à palettes \(S = (1.12 \pm 0.01)\) \si{\liter \per \second}. La pression limite \(P_0 = P_\textrm{lim,1}\) est obtenue au bout de 15 \si{\minute} de pompage et vaut \(P_0 = (1.0 \pm 0.5) \times 10^{-2}\) \si{\milli\bar}.
\end{minipage}

\begin{minipage}{\textwidth}
    \begin{wrapfigure}{R}{0.5\textwidth}
        \centering
        \includegraphics[width=\linewidth]{figures/cinetique_diffusion.pdf}
        \caption{Pression dans la chambre pour la pompe à diffusion en fonction du temps}
        \label{fig:cinetique_diffusion}
    \end{wrapfigure}
    
    \paragraph*{Débit de la pompe à diffusion}
    La pompe à diffusion permet d'atteindre le vide secondaire dans la chambre. L'évolution de la pression en fonction du temps est donnée sur la \autoref{fig:cinetique_diffusion}. La régression linéaire a été réalisée sur les points entre \(t = 2\) \si{\second} et \(t = 7\) \si{\second}. La pression de référence \(P_1 = (0.049 \pm 0.005)\) \si{\milli\bar} à été utilisée. L'\autoref{eq:cinetique} donne alors un débit effectif pour la pompe à diffusion de \(D = (9 \pm 1)\) \si{\liter \per \second}. Au bout de 15 \si{\minute}, une pression limite \(P_0 = P_\textrm{lim,2} = (1.2 \pm 0.5) \times 10^{-5}\) \si{\milli\bar} est atteinte.
\end{minipage}






\paragraph*{Changement d'état de l'azote}
\begin{figure}
    \centering
    \includegraphics[width=0.6\textwidth, height=7cm]{figures/etats_azote_all_data.pdf}
    \caption{Points de cohabitation de deux états de l'azote}
    \label{fig:alldata}
\end{figure}

Pour la deuxième partie de l'expérience l'étude des changements d'état de l'azote à volume constant a été effectuée. Cinq ensembles de mesure ont été pris successivement en arrêtant le pompage à différents moments. Il est important de noter que l'installation disponible au moment de l'exécution de l'expérience manquait d'un agitateur utile pour homogénéiser le mélange. A plusieurs moments durant les mesures il a donc été possible d'observer une cohabitation des trois états même en dehors du point triple, les conditions de température n'étant pas les même dans tout le volume. \\
Une première mesure de la courbe de sublimation avec arrêt du pompage après solidification complète et refroidissement en dessous de 55 \si{\kelvin} a été effectuée. Celle-ci s'est poursuivie par une mesure de la courbe de fusion sans disparition complète de l'azote solide. Les deux autres mesures de la courbe de sublimation se sont poursuivies par un palier inattendu entre 121 et 126 \si{\milli \bar}. La fin de ce palier a été caractérisé à deux reprises par un saut important dans les données (\(\Delta T = 15\) \si{\kelvin} en 12 \si{\second}) et a été suivi par le relevé d'une courbe de vaporisation. Deux autres mesures ont été effectuées, l'une après la solidification complète de l'azote liquide afin de relever une courbe de fusion supplémentaire et l'autre juste au début de la solidification afin de relever une autre courbe de vaporisation. L'ensemble des mesures est présenté dans le diagramme (\(P\), \(T\)) en \autoref{fig:alldata}. \\
\\
Afin d'obtenir des courbes de changement d'état plus claires les données obtenues ont été sélectionnées. La courbe de sublimation non-polluée par un palier a été gardée. Le palier a été entièrement retiré et les points des courbes de fusion à des températures inférieures au point triple (65 \si{\kelvin}), qui correspondent à des variations du système au point triple, ont également été supprimés. Ce traitement des données a permis d'obtenir la \autoref{fig:cleandata}.

\paragraph*{Point triple et \(T_\textrm{min}\)}
La \autoref{fig:cleandata} permet d'estimer le point triple à l'intersection des trois courbes de changement d'état. Le point triple relevé sur le diagramme est: \(T_\textrm{0} = (65 \pm 2)\)\si{\kelvin} et \(P_\textrm{0} = (150 \pm 20)\)\si{\milli \bar}. Cela permet de calculer \(T_\textrm{min}\) avec \autoref{eq:Tmin} en connaissant \(R = 8.31\) \si{\joule \per \mole \per \kelvin} et \(L_\textrm{S} = 6.28 \times 10^{3}\) \si{\joule \per \mole} \cite{Azote}.\\
La température minimale atteignable par pompage sur de l'azote avec le système étudié est: \(T_\textrm{min} = (36 \pm 2) \) \si{\kelvin}.

\paragraph*{Entrée d'azote sur azote solide}
Après pompage au-delà de la solidification complète de l'azote il est possible d'ouvrir la vanne vers le réservoir d'azote liquide en \autoref{fig:montagepointtriple}. L'observation faite est que l'azote solide présent dans le dewar principal se liquéfie rapidement pour être en équilibre avec l'azote liquide du réservoir présent en quantité plus importante.



\begin{figure}
    \centering
    \includegraphics[width=0.6\textwidth, height=7cm]{figures/etats_azote_clean.pdf}
    \caption{Courbes de changement d'état de l'azote}
    \label{fig:cleandata}
\end{figure}




