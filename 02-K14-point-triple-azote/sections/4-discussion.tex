\section{Discussion}

\paragraph{Cinétique de pompage}
Les cinétiques de pompage de deux pompes ont été déterminées, cela permet désormais de connaître précisément l'évolution de la pression dans un volume donné pendant que celui-ci est vidé par ces pompes. De plus, la connaissance de la pression limite du système avec ces pompes permet des applications plus concrètes encore car connaître la pression d'équilibre d'un système est très utile et à d'ailleurs été utilisé dans la deuxième partie de cette expérience pour déterminer la température minimale atteignable par le système étudié.

La pompe à palettes possède un petit temps de \(t = 4\) \si{\second} de mise en route, ne permettant pas de mesurer le debit dès le début du pompage. Pour améliorer cette latence, la pompe aurait pu être mise en route un peu avant l'ouverture de la vanne principale.

Le calcul du débit de la pompe à diffusion n'est pas très précis (environ 10\% de variation) en raison du changement brusque de débit après \(t = 7\) \si{\second} et du petit temps (\(t = 2\) \si{\second}) de mise en route après l'ouverture de la vanne. De plus, le manomètre change pendant la mesure, passant de la jauge de Piranni à la jauge de Penning, engendrant 1 \si{\second} d'absence de mesures, pendant la partie ou le débit est plus ou moins linéaire. Il faudrait donc améliorer le montage pour pouvoir mesurer la sortie des deux jauges en parallèle pour obtenir de meilleurs résultats.


\paragraph*{Point triple de l'azote}
Le point triple de l'azote ainsi que ses courbes de changement d'état ont été mesurés. La connaissance du point triple d'un corps présente un intérêt pour la connaissance fondamentale ainsi que pour des applications plus concrètes, par exemple comme constante pour résoudre des systèmes et des intégrations. La connaissance des courbes de changement d'état est cruciale pour connaître l'évolution de nombreux systèmes thermodynamiques et présente donc de nombreuses applications industrielles, par exemple il peut être dangereux pour un système à basse températures et pressions de voir l'azote gazeux se sublimer à la surface d'un appareil, d'une espèce chimique à étudier...

Il est intéressant de noter que la température minimale obtenue par calcul (\(T_\textrm{min} = (36 \pm 2) \) \si{\kelvin}) est tout de même bien supérieure à la plus petite température mesurée au cours de l'expérience: \(T_\textrm{exp,min} = 55\) \si{\kelvin}. Cela indique que cette température minimale \(T_\textrm{min}\) nécessiterait probablement un long temps de pompe pour être atteinte.

La réalisation de cette expérience a cependant été gênée par certains défauts dans l'installation. En effet le système disponible pour pomper sur un dewar d'azote ne possédait pas d'agitateur. Cela a empêché une bonne homogénéisation de la température au sein de notre système ce qui a probablement impacté l'allure de nos courbes et les valeurs finales obtenues. Un élément très frappant a été la formation à plusieurs reprises d'un important bloc d'azote solide sur toute la surface de l'azote liquide au lieu de le voir bien réparti en cristaux à l'intérieur. Le palier inattendu qui a pu être observé à deux reprises de manière significative est probablement une conséquence de cette non-homogénéisation. Une hypothèse probable est que ce morceaux de solide englobait la tête de mesure de la sonde platine ce qui faussait les mesures. La fonte complète du solide correspond d'ailleurs à la chute importante de température à la fin du palier (\(\Delta T = 15\) \si{\kelvin} en 12 \si{\second}) ce qui correspondrait dans cette hypothèse à la mise en contact de la sonde avec l'azote liquide à une température différente.

Un autre défaut de l'expérience est la difficulté à relever des mesures (\(P\), \(T\)) exactement en simultané. En effet la collaboration au sein du binôme ou la prise d'une vidéo permet d'atteindre cela mais une installation électronique permettant de prendre plus de points et avec une simultanéité plus fiable pourrait être utile.
