\section{Discussion}

\paragraph{Cinétique de pompage}
La pompe à diffusion possède un petit temps de \(t = 4\) \si{\second} de mise en route, ne permettant pas de mesurer le debit dès le début du pompage. Pour ameillorer cette latence, la pompe aurait du être allumée un peu avant l'ouverture de la vanne principale.

Le calcul du débit de la pompe à diffusion n'est pas très précis (environ 10\% de variation) en raison du changement brusque de débit après \(t = 7\) \si{\second} et du petit temps (\(t = 2\) \si{\second}) de mise en route après l'ouverture de la vanne. De plus, le manomètre change pendant la mesure, passant de la jauge de Piranni à la jauge de Penning, engendrant 1 \si{\second} d'absence de mesures, pendant la partie ou le débit est plus ou moins linéaire. Il faudrait donc ameillorer le montage pour pouvoir mesurer la sortie des deux jauges en parallèle pour obtenir de meilleurs resultats.
