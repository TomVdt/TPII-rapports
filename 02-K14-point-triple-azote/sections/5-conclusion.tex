\section{Conclusion}

Cette expérience a permis d'étudier la vitesse de pompage dans une installation à vide. Bien que le débit effectif de la pompe à diffusion soit plus grand d'un facteur 10 que celui de la pompe à palettes, ce débit n'est pas maintenu tout au long du pompage. L'étude de ce débit pourrait permettre de créer des pompes plus efficaces et moins volumineuses pour les applications du vide comme la conservation longue durée des aliments.

Il a aussi été possible de mesurer, à l'aide partiellement des nouvelles connaissances sur la cinétique du pompage, différentes caractéristiques d'un système thermodynamique. Cela permet une meilleure connaissance des caractéristiques physiques de l'azote. La connaissances des conditions d'équilibres de ce système pourrait également avoir des applications plus pratiques comme par exemple la garantie de certaines conditions de pression et température dans l'industrie.