\section{Théorie}

La mise sous vide d'une installation nécessite différentes installations de pompage et de mesure (manomètre mécanique de Bourdon puis jauges Pirani et Penning) selon le niveau de vide souhaité. Le vide primaire (de 1000 à \(10^{-3}\) \si{\milli \bar}) s'obtient à l'aide d'une pompe à palette accompagnée d'un mécanisme de lest d'air à certaines pressions. Ce système diminue le taux de commpression et empêcher la condensation des vapeurs présentes dans le volume qui pourraient dégrader la lubrification au sein de la pompe. La pompe à diffusion est une pompe secondaire qui vaporise de l'huile et oriente les vapeurs ainsi produites pour rediriger les gazs du volume vers une pompe primaire permettant d'atteindre un vide poussé. \\
Pour chaque pompe et chaque système sous vide il existe une pression limite \(P_\textrm{lim}\) qui correspond à la pression d'équilibre minimale que la pompe permet d'atteindre en prenant en compte le dégazage et les fuites. \\
Pour déterminer le débit effectif \(S\) d'une pompe qui permet de prédire le comportement du système durant le pompage l'équation suivante est connue:
\begin{equation}
    \ln(\frac{P_\textrm{2} - P_\textrm{lim}}{P_\textrm{1} - P_\textrm{lim}}) = -\frac{S}{V}t
    \label{eq:cinetique}
\end{equation}
Avec \(t\) le temps nécessaire pour passer de \(P_\textrm{1}\) à \(P_\textrm{2}\), \(P_\textrm{1} > P_\textrm{2}\) et \(V\) le volume à vider \cite{notice}. \\
\\
Les changements d'état d'un corps pur peuvent se représenter sur un diagramme (\(P\), \(T\)) comprenant trois courbes: de sublimation, de fusion, de vaporisation. Ces courbes représentent les points où se font les changements d'état quand ils se déroulent à pression et température constante (\(V \neq constante\)). Il est possible d'obtenir ces courbes autrement en maintenant \(V\) constant et en permettant un réchauffement du système. La position d'équilibre entre les deux états de la matière va donc forcer le système à rester sur les courbes de changement d'état. \\
Une particularité des diagrammes de changement de phase (\(P\), \(T\)) est la présence d'un point triple (\(P_\textrm{0}\), \(T_\textrm{0}\)) à l'intersection des trois courbes. Celui-ci représente les conditions de température et pression où les trois états de la matière peuvent coexister pour ce corps pur.\\
\\
Connaissant les caractéristiques de la pompe et des constantes thermodynamiques il est possible de calculer la température minimale qu'un système contenant un corps pur, ici l'azote, et étant pompé peut atteindre:
\begin{equation}
    T_\textrm{min} = (\frac{1}{T_\textrm{0}} - \frac{R}{L_\textrm{S}}\log(\frac{P_\textrm{min}}{P_\textrm{0}}))^{-1}
    \label{eq:Tmin}
\end{equation}
Avec \(P_\textrm{min} = P_\textrm{lim}\) la plus basse pression que la pompe peut atteindre, \(R\) la constante des gazs parfaits, \(L_\textrm{S}\) la chaleur latente de sublimation de l'azote et le point triple (\(P_\textrm{0}\), \(T_\textrm{0}\)) de l'azote. 
