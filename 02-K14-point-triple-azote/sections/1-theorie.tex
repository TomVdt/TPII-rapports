\section{Théorie}



La mise sous vide d'une installation nécessite différentes installations de pompage et de mesure (manomètre mécanique de Bourdon puis jauges Pirani et Penning) selon le niveau de vide souhaité. Le vide primaire (de 750 à \(10^{-3}\) Torr) s'obtient à l'aide d'une pompe à palette. Un lest d'air est nécessaire pour diminuer le taux de commpression et empêcher la condensation des vapeurs présentes dans le volume ce qui pourrait dégrader la lubrification au sein de la pompe. Pour le vide secondaire une pompe à diffusion peut être utilisée, cette pompe vaporise de l'huile qui redirige ensuite les gazs du volume vers une pompe primaire permettant d'atteindre un vide poussé (ici \(\sim 10^{-5}\) mbar). \\
Pour chaque pompe et chaque système sous vide il existe une pression limite \(P_{lim}\) qui correspond à la pression d'équilibre minimale que la pompe permet d'atteindre en prenant en compte le dégazage et les fuites. \\
Pour déterminer le débit effectif \(S\) d'une pompe qui permet de prédire le comportement du système durant le pompage l'équation suivante est connue:
\begin{equation}
    \ln(\frac{P_2 - P_{lim}}{P_1 - P_{lim}}) = -\frac{S}{V}t
\end{equation}
Avec \(t\) le temps nécessaire pour passer de \(P_1\) à \(P_2\) avec \(P_1 > P_2\) et \(V\) le volume à vider \cite{notice}. \\
\\
Les changements d'état d'un corps pur peuvent se représenter sur un diagramme (\(P\), \(T\)) comprenant trois courbes: de sublimation, de fusion, de vaporisation. Ces courbes représentent les points où se font les changements d'état qui se déroulent à pression et température constante (\(V \neq constante\)). 
