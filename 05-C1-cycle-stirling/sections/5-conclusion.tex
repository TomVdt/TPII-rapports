\section{Conclusion}

Lors de cette expérience, le cycle de Stirling à été étudié en mode moteur et en mode frigorifique et son rendement et sa réponse à la vitesse de rotation ont été observés. En mode moteur, une forte différence entre le rendement théorique du cycle et le rendement expérimental a été observé. Les différentes méthodes de mesure du rendement ont permis de mettre en évidence l'importance des pertes dans l'application réelle des modèles théoriques. Le moteur Stirling utilisé pour cette expérience a un rendement trop faible pour des applications réelles, à moins de 5\% comparé aux 15\% dans les voitures à moteur à combustion roulant en ville \cite{voiture}. La dépendance entre la vitesse de rotation et le rendement est cruciale dans la conception de moteurs efficaces sur une grande gamme de vitesses, ce qui est nécessaire pour des application automobiles par exemple.
