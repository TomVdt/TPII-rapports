\section{Introduction}
La réalisation du premier moteur de Stirling en 1842 par James Stirling a eu lieu pendant la période de la domination des machines à vapeur \cite{histoire-stir}. La création de ce nouveau moteur arriva avant que son analyse thermodynamique formelle soit possible avec les outils théoriques de l'époque, la première analyse fut réalisée seulement en 1871 par Gustav Schmidt \cite{histoire-schmidt}. Ainsi il fut nécessaire d'estimer ses performances de manière empirique pour son utilisation efficace.

Les expériences présentées sont donc des méthodes pour caractériser différents aspects et fonctionnements du cycle de Stirling. Les résultats montrent tout d'abord le rendement en fonctionnement moteur obtenu de trois manières différentes. La réponse de ce fonctionnement moteur à différentes vitesses de rotation est également caractérisée. La dernière expérience consiste enfin à étudier le rendement en fonctionnement frigorifique du cycle.