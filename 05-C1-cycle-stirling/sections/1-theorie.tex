\section{Théorie}






\paragraph*{Cycle de Stirling en pompe à chaleur}
Un moteur de Stirling peut également fonctionner en mode pompe à chaleur ou machine frigorifique, la seule différence étant l'intérêt dans l'apport ou l'extraction de chaleur. Il s'agit dans ce cas d'entraîner le cycle à l'aide d'un moteur extérieur. Selon le sens de rotation de l'arbre les flux de chaleurs produits par la pompe à chaleur seront dans des sens différents, dans le même sens pour une même rotation et dans le sens inverse dans le cas contraire. TODO: dire que équation est la même à un signe près

Les valeurs du rendement se basent sur le même concept mais les flux d'énergies considérés ne sont plus les même. Dans le cas d'un cycle frigorifique, la puissance fournie est la puissance du moteur d'entraînement \(P_{moteur}\) et la grandeur d'intérêt est le flux de chaleur évacué \(\phi\). Le rendement est donc donné par l'équation:
\begin{equation}
    \rho = \frac{\phi}{P_{moteur}}
    \label{eq:rend-frigo}
\end{equation}