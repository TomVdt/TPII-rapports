\section{Théorie}

\begin{minipage}{\linewidth}
    \begin{wrapfigure}{R}{0.55\linewidth}
        \includegraphics*[width=\linewidth]{figures/cycle-stirling.png}
        \caption{Diagramme \((P,V)\) du cycle de Stirling}
        \label{fig:cycle-pv}
    \end{wrapfigure}

    \paragraph*{Etude du cycle de Stirling}
    Le cycle de Stirling est un type de machine thermique ditherme, qui prend donc un flux de chaleur \(\phi_2\) d'une source de chaleur à température \(T_2\) et produit une puissance mécanique \(P_m\) en expulsant également un flux de chaleur \(\phi_1\) vers sa source froide à \(T_1\). Le cycle de Stirling en particulier est composé de 4 transformations, 2 isothermes et 2 isochores entre \(V_1\) et \(V_2\) tel qu'illustré en \autoref{fig:cycle-pv}. Le rendement théorique de ce cycle est obtenu en étudiant les échanges de chaleur et de travail sur ces transformations. Tout d'abord, sur les transformations isochores \hbox{\(W_{2\to3} = W_{4\to1} = -p \, dV = 0\)}. De plus le cycle de Stirling utilise un régénérateur permettant de restituer la chaleur de \(2\to3\) pendant \(4\to1\), ainsi \(Q_{2\to3} = -Q_{4\to1}\).
\end{minipage}

Cela permet de déterminer \(Q_{1\to2} = Q_2\) et \(W_{cycle} = W = -Q_1 - Q_2\) avec \(Q_1 = Q_{3\to4}\), car il s'agit d'un cycle avec une variation d'énergie interne totale nulle. Cela donne pour \(W\) en considérant les gazs du système comme parfait:
\begin{equation}
    W = - \int_{V_1}^{V_2} \frac{nRT_2}{V} \, dV - \int_{V_2}^{V_1} \frac{nRT_1}{V} \, dV = nR\ln\left(\frac{V_2}{V_1}\right) (T_1 - T_2)
    \label{eq:Wstir}
\end{equation}
avec \(n\) la quantité de matière du système et \(R\) la constante des gazs parfaits.
De plus le cycle garantit que \(\Delta S = 0\) donc \(S_{1\to2} = S_{3\to4}\) avec les flux de chaleurs des isochores se compensant également. Cela permet d'obtenir:
\begin{equation}
    Q_2 = \int_{S_1}^{S_2} T_2 \, dS = T_2 S_{1\to2} = -\frac{T_2}{T_1} Q_1 = \frac{T_2}{T_1} (W + Q_2)
\end{equation}
\begin{equation}
    \Rightarrow Q_2 = W \frac{T_2}{T_1 - T_2}
    \label{eq:Q2stir}
\end{equation}
Finalement le rendement est obtenu par:
\begin{equation}
    \eta = \left|\frac{W_{cycle}}{Q_2}\right| = \frac{T_2 - T_1}{T_2} = 1 - \frac{T_1}{T_2}
    \label{eq:rend-theorie}
\end{equation}
qui est le rendement d'un cycle de Carnot, le cycle de Stirling est donc en théorie réversible \cite{cours-thermo}.



\paragraph*{Cycle de Stirling en pompe à chaleur}
Un moteur de Stirling peut également fonctionner en mode pompe à chaleur ou machine frigorifique, la seule différence étant l'intérêt dans l'apport ou l'extraction de chaleur. Il s'agit dans ces cas d'entraîner le cycle à l'aide d'un moteur extérieur. Selon le sens de rotation de l'arbre les flux de chaleurs produits par la pompe à chaleur seront dans des sens différents, dans le même sens pour une même rotation et dans le sens inverse dans le cas contraire. Dans le cas du même sens de rotation les valeurs des \(W\) et \(Q_2\) obtenus sont les opposés de ceux obtenus dans les \autoref{eq:Wstir} et \autoref{eq:Q2stir}.

Le calcul du rendement vient du même concept mais les flux d'énergies considérés ne sont plus les même. Dans le cas d'un cycle frigorifique, la puissance fournie est la puissance du moteur d'entraînement \(P_{moteur}\) et la grandeur d'intérêt est le flux de chaleur évacué \(\phi\). Le rendement est donc donné par l'équation:
\begin{equation}
    \eta = \frac{\phi}{P_{moteur}}
    \label{eq:rend-frigo}
\end{equation}