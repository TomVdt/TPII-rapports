\section{Discussion}

\paragraph{Erreurs de mesure} Le montage permettant d'effectuer les mesures permettait une très haute précision pour déterminer les grandeurs recherchées. Cependant, la majeure partie des erreurs est contenue dans la mesure des dimensions des échantillons. En effet, malgré la précision offerte par le matériel fourni (micromètre précis à \(10^{-2}\) \si{\milli\meter}), la variabilité de l'échantillon ainsi que la manipulation augmentent grandement l'incertitude sur la valeur obtenue, qui s'amplifie rapidement pour le calcul de la masse volumique.

\paragraph{Mesures à température variable} L'accumulation de bruit de fond lors de la mesure du module de Young et de la capacité d'amortissement indique potentiellement une nécessité de recallibrer l'appareil en milieu de mesure, afin de limiter l'accumulation de l'erreur systématique. De plus, il y avait une forte variabilité dans les valeurs de \(Q^-1\) pour un \(T\) proche, indiquant une certaine incertitude sur la valeur donnée par le programme. Cependant, sans avoir accès aux données d'amplitude des oscillations, il est difficile de donner une marge d'erreur à la valeur sortie. La nécessité d'ajuster la position de l'électrode par la vis micromètrique était aussi probablement une source d'imprécision, puisque la position était changée par petits coups et non de facon continue. L'électrode à touché l'échantillon d'acier doux quelques fois lors des ajustements, mais cela ne semble pas avoir d'effet significatif sur les résultats.

\paragraph{Comparaison aux valeurs de référence} Les résultats obtenus pour le module de Young \(E\) ont été comparés aux valeurs de références
