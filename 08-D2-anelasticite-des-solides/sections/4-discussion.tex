\section{Discussion}

\paragraph{Erreurs de mesure} Le montage permettant d'effectuer les mesures permettait une très haute précision pour déterminer les grandeurs recherchées. Cependant, la majeure partie des erreurs est contenue dans la mesure des dimensions des échantillons. En effet, malgré la précision offerte par le matériel fourni (micromètre précis à \(10^{-2}\) \si{\milli\meter}), la variabilité de l'échantillon ainsi que la manipulation augmentent grandement l'incertitude sur la valeur obtenue, qui s'amplifie rapidement pour le calcul de la masse volumique.

\paragraph{Mesures à température variable} L'accumulation d'erreurs systématiques lors de la mesure du module de Young et de la capacité d'amortissement en fonction de la température indique potentiellement une nécessité de recalibrer l'appareil en milieu de mesure pour limiter cet effet. De plus, il y avait une forte variabilité dans les valeurs de \(Q^-1\) pour un \(T\) proche, indiquant une certaine incertitude sur la valeur donnée par le programme. Cependant, sans avoir accès aux données d'amplitude des oscillations, il est difficile de donner une marge d'erreur à la valeur sortie. La nécessité d'ajuster la position de l'électrode par la vis micromètrique était aussi probablement une source d'imprécision, puisque la position était changée par petits coups et non de facon continue. L'électrode à touché l'échantillon d'acier doux quelques fois lors des ajustements, mais cela ne semble pas avoir d'effet significatif sur les résultats.

\paragraph{Comparaison aux valeurs de référence} Les résultats obtenus pour le module de Young \(E\) ont été comparés aux valeurs de références, ce qui donne une erreur relative très faible (1\% et 6\%) pour le titane et le laiton, et un peu plus grande (14\%) pour l'acier doux. Il est possible que la composition de l'échantillon d'acier doux soit différente que celle indiquée dans la source, et que ce soit donc la cause d'une plus grande différence. Cette méthode de mesure semble donc fonctionner avec une précision plutot élevée, ou au moins plus élevée que les méthodes proposées pour le TP D1.


\paragraph{Analyse de la relaxation de Snoek} La relaxation de Snoek pour l'acier doux a donc été bien étudiée et plusieurs informations peuvent en être rétirées. Tout d'abord la théorie a remarquablement bien prédit le comportement expérimental notamment entre l'\autoref{eq:Q_inv_T} et la \autoref{fig:acier_doux_temp_adjusted}. Cela a permis d'obtenir des résultats cohérents malgré une certaine imprécision dans l'extraction des valeurs de $\Delta / 2$, $T_P$, $T_1$ et $T_2$ notamment. La grande proximité de la théorie avec l'observation expérimentale et le grand nombre de mesures relevées permet de déduire que les résultats pour l'énergie de diffusion et la concentration en défauts dans le cristal sont proches des valeurs réelles malgré une incertitude plus importante sur certaines valeurs individuelles.