\section{Introduction}
Les déformations des matériaux sont des phénomènes importants à analyser pour de nombreuses applications dans la construction, l'ingénierie ou l'expérimentation. La déformation anélastique qui peut accompagner les déformations élastiques apporte plusieurs opportunités et informations. Cette déformation peut avoir des applications malgré sa faible amplitude comme par exemple pour amortir des vibrations non souhaitées \cite{damp-damp}. Elle est également riche en information car elle donne des indices sur la composition du matériau et notamment sur les défauts qu'il contient \cite{notice}.

Cette expérience consistera donc à analyser la relaxation anélastique dans 3 matériaux différents l'acier doux, le laiton et le titane. Cela permettra de déterminer leur module de Young pour la déformation élastique mais également leur capacité à amortir les vibrations ou encore certaines caractéristiques microscopiques liées à la relaxation anélastique de Snoek.