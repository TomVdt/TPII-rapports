\section{Théorie}

Le sujet de cette expérience est la relaxation anélastique. Ce phénomène a lieu quand un matériau est soumis à une contrainte $\ligma$ de faible intensité. Une première déformation instantanée, dite élastique, $\varepsilon_e = J_u \ligma$ est observée avec $J_u$ la complaisance non relaxée. Elle peut aussi être décrite par $\ligma = E\varepsilon_e$ avec $E = 1/J_u$ le module de Young. Ensuite la déformation anélastique $\varepsilon_a$ augmente progressivement de $0$ vers une valeur d'équilibre $\varepsilon_a^\infty$ telle que:
\begin{equation}
    \varepsilon_a(t) = \varepsilon_a^\infty \left(1-e^{-\frac{t}{\tau}}\right)
\end{equation}
\begin{equation}
    \varepsilon = \varepsilon_e + \varepsilon_a^\infty = J_r \ligma
\end{equation}
avec $\tau$ le temps de relaxation anélastique et $J_r$ la complaisance relaxée. Il est utile également de définir l'intensité de relaxation: $\Delta = \frac{J_r - J_u}{J_u}$

En appliquant une contrainte harmonique $\ligma = \ligma_0 \exp(i\omega t)$ de pulsation $\omega$ au matériau sa réponse en déformation sera elle aussi périodique $\varepsilon = \varepsilon_0\exp(i(\omega t - \delta))$ avec un déphasage $\delta$. Ce déphasage est relié au frottement intérieur du matériau $Q^{-1}$ et dans le cas où $(J_r - J_u) \ll J_u$:
\begin{equation}
    Q^{-1} = \tan(\delta) = \Delta\frac{\omega\tau}{1+\omega^2\tau^2}
\end{equation}
ce qui permet de déduire que $Q^{-1}$ possède un maximum en fonction de $\omega\tau$ en $\omega\tau = 1$. En sachant qu'il est possible d'écrire $\tau(T) = \tau_0 \exp(\frac{H}{k_bT})$ avec $H$ l'enthalpie ou énergie de diffusion, $k_b = 1.38 \times 10^{-23}$ \si{\joule \per \kelvin} la constante de Boltzmann et $T$ la température cela donne:
\begin{equation}
    Q^{-1}(T) = \frac{\Delta}{2 \mathrm{ch}\left(\frac{H}{k_b}\left(\frac{1}{T_P}-\frac{1}{T}\right)\right)}
    \label{eq:Q_inv_T}
\end{equation}
avec $T_P$ la température du pic c'est à dire la température telle que $\omega\tau(T_P) = 1$.

L'\autoref{eq:Q_inv_T} permet également de trouver une formule pour la valeur de $H$. En connaissant $T_1$ et $T_2$ les deux valeurs telles que $Q^{-1}(T_1) = Q^{-1}(T_2) = \frac{\Delta}{4}$, c'est-à-dire la mi-hauteur du pic, il est possible de déduire l'expression:
\begin{equation}
    H = \frac{2.633 k_b}{\frac{1}{T_1} - \frac{1}{T_2}}
    \label{eq:H_mi_hauteur}
\end{equation}
en ayant choisi $T_1 < T_2$.


La relaxation anélastique est la réalisation macroscopique du réarrangement microscopique des défauts du matériau élastique. Dans le cadre d'une contrainte uniaxiale cela donne \hbox{$\varepsilon_a = \Delta\lambda\left(c_1 - c_0/3\right)$} avec $c_0$ la concentration atomique des défauts interstitiels, $c_1$ la concentration de ceux fromant un dipole dans l'axe de la contrainte et $\Delta\lambda$ la variation de déformation due aux mouvements des défauts sous la contrainte. Cela donne une nouvelle relation:
\begin{equation}
    \Delta = \frac{2}{9}\frac{c_0 v_0}{J_u k_b T_P}\Delta\lambda^2
    \label{eq:delta_defauts}
\end{equation}
avec $v_0$ le volume atomique des défauts interstitiels.
Cela permet d'avoir une expression pour la concentration des interstitiels:
\begin{equation}
    c_0 = \frac{9}{2}\frac{\Delta J_u k_b T_P}{v_0 \Delta \lambda^2}
    \label{eq:c_0}
\end{equation}