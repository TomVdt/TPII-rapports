\section{Conclusion}

Lors de cette expérience la relaxation anélastique de plusieurs échantillons de matériaux différents a été étudié, sous plusieurs conditions de température notamment. Ainsi leur module de Young \(E\) et leur capacité d'amortissement \(Q^{-1}\) ont été calculés. Cela donne des informations importantes sur le comportement sous la contrainte de ces matériaux. Entre les trois échantillons testés, l'acier doux avait la plus grande capacité d'amortissement. Ce résultat est donc cohérent avec l'utilisation de l'acier dans la construction comme matériaux principal pour la résistance parasysmique \cite{acier-construction}. Les méthodes utilisées ici permettent également d'obtenir des informations sur la structure microscopique des matériaux. La concentration en défauts interstitiels indique la pureté du cristal et comparer plusieurs échantillons du même matériau avec l'expérience présentée ici pourra permettre d'évaluer leurs puretés respectives.
