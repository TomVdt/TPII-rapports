\section{Conclusion}

\begin{itemize}
\item Résumé, dense, du travail. Réponse claire à l'introduction.
\item \textit{Objectifs atteints}?
\item La conclusion doit remettre vos résultats dans le contexte mentionné dans l'introduction et montrer en quoi vos mesures sont importantes/utiles/intéressantes et applicables (dans le projet). N'hésitez pas à remettre les valeurs numériques d'intérêt dans le texte pour appuyer vos déductions.
\item Il ne faut pas commencer par "En conclusion, … ", c'est évident du titre de la section
\end{itemize}