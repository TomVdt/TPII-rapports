\section{Discussion}

\paragraph{Méthode statique}
Les résultats obtenus pour le module de cisaillement avec la méthode statique sont plutôt proches des valeurs de référence, avec au maximum 30\% d'écart relatif avec la valeur obtenue. La différence en précision pour l'échantillon de magnésium pourrait s'expliquer par au moins deux facteur. Le premier facteur est le diamètre du cylindre, qui était deux fois plus grand que les autres échantillons. Cela donnait probablement plus de rigidité à l'échantillon, et donc une réaction moins forte qu'attendue au changement de longueur de l'échantillon. Deuxièmement, le choix d'une longueur plutôt petite ($L=(25.0\pm0.1)$ \si{\centi\meter}) pour cet échantillon a possiblement diminué l'effet des masses, ce qui a pu être observé lors de la collecte de données: l'ajout d'une masse ne signifiait pas toujours l'augmentation de la déviation $\delta$ du laser. Une ameilloration à ce montage pourrait s'inspirer du montage dynamique: utiliser un système de capteur de lumière pour augmenter la précision de lecture de la déviation.

Cette méthode donne aussi plus de contrôle sur les paramètres, qui pouvaient être changés uns part uns, contrairement à la méthode dynamique, où une fois le pendule à torsion lancé, le contrôle sur la déviation par exemple est perdu.

\paragraph{Méthode dynamique}
Cette méthode résulte en des valeurs du module de cisaillement bien plus éloignées de valeurs de référence, avec un écart relatif de plus de 45\% pour chaque échantillon. Lors des mesures, il a été observé que le dispositif n'empêchait pas le mouvement dans d'autres directions, notamment sur les cotés. Le signal enregistré était donc la superposition d'une rotation (torsion de l'échantillon) et des vibration sur les cotés du disque d'inertie. Cela a pour effet de changer la distance entre le miroir et le capteur, influencant ainsi les mesures. Ce phénomène est particulièrement visible pour une des mesures réalisées sur le magnésium, comme il est possible de voir dans la \autoref{fig:dynamique_magnesium} ou la \autoref{fig:dynamique_magnesium_feur} en \autoref{sec:figsup}. De plus, ce dispositif est bien plus sensible aux petits changements ou aux chocs accidentels. L'effet d'un leger choc est visible aussi sur la \autoref{fig:dynamique_magnesium_feur}, à $t=40$ \si{\second}.

Cette configuration du dispositif est donc bien moins précise pour déterminer le module de cisaillement de différents matériaux que la dispositif statique. Un avantage potentiel est qu'il nécessite moins de matériau pour effectuer une mesure, puisque les échantillons doivent être sous forme de fil. Cela est cependant potentiellement à double tranchant, en raison de matériaux moins solides qui se briseraient sous cette forme.

