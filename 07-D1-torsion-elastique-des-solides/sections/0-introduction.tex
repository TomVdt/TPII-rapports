\section{Introduction}

L'étude des propriétés des matériaux est essentielle dans le domaine de la construction, notamment pour la resistance aux tremblements de Terre. Le Japon a subit dans les derniers 10 ans plus de 60 tremblements de Terre de magnitude 6 ou plus, le rendant un des pays les plus exposés aux tremblements de Terre au monde \cite{japan_shakey}. Afin que les batiments, ainsi que leurs occupants, survivent, il est nécessaire que les déformations subies par la structure des batiments soient élastiques, permettant de répartir la force du séisme sur tout le batiment.

Le but de cette expérience est d'étudier l'élasticité de plusieurs solides, le magnesium, laiton, et l'acier. Plus précisément, leur module de cisaillement sera déterminé par une méthode statique et une méthode dynamique. Les valeurs obtenues par les deux méthodes seront comparées entre elles et à des valeurs de référence, puis une discussion sur les avantages et incovénients de chaque méthode aura lieu.