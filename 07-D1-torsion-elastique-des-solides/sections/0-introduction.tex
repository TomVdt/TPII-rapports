\section{Introduction}

L'étude des propriétés des matériaux a de nombreuses applications notamment dans le domaine de la construction, par exemple pour construire des bâtiments résistants aux tremblements de terre. Le Japon a subit dans les 10 dernières années plus de 60 tremblements de terre de magnitude 6 ou plus, en faisant un des pays les plus exposés aux tremblements de terre au monde \cite{japan_shakey}. Afin que les bâtiments, ainsi que leurs occupants, survivent à ceux-ci il est nécessaire que les déformations subies par leur structure soient élastiques, permettant ainsi de répartir la force du séisme dans tout le bâtiment.

Le but des expériences présentées est d'étudier l'élasticité de plusieurs solides, le magnesium, le laiton et l'acier. Plus précisément, leur module de cisaillement sera déterminé par une méthode statique et une méthode dynamique. Les valeurs obtenues par les deux méthodes seront comparées entre elles et à des valeurs de référence, puis une discussion sur les avantages et inconvénients de chaque méthode aura lieu.