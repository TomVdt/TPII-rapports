\section{Conclusion}

Lors de cette expérience, deux méthodes pour mesurer le module de cisaillement $G$ ont été comparées, sur trois échantillons de métaux chacun. Les valeurs obtenues pour $G$ ont ensuite été comparées aux valeurs de référence, afin d'évaluer la méthode. Les avantages et inconvénient de chaque méthode ont été donnés. Dans le cadre de résistance aux séismes, le choix d'un matériaux avec un module de cisaillement peut se prouver critique: s'il est trop faible, alors le batiment ne tient pas debout, alors que s'il est trop élevé, la structure devient trop rigide pour faire face à un tremblement de Terre et finir par céder au lieu de se tordre pour absorber le choc. De plus, le type de tremblement de Terre peut modifier la réponse de la structure d'un batiment \cite{japan_shaky_stiffness}. Il est donc essentiel de connaitre les propriétés élastique d'un matériaux afin de s'en servir le mieux possible!