\section{Résultats}

\paragraph*{Pouvoir calorifique théorique de l'éthanol}
Conformément aux \autoref{eq:chaleur_réaction} et \autoref{eq:ethanol_combustion} il est possible de calculer le pouvoir calorifique de l'éthanol, c'est-à-dire la chaleur que sa combustion libère. Pour ce faire il est nécessaire de connaître sa chaleur de formation \(\Delta H_{l} = (-276 \pm 2)\) \si{\kilo\joule\per\mol} et sa masse molaire \(\rho = (46.07 \pm 0.01)\) \si{\gram\per\mol} \cite{ethanol-values}. Les valeurs des chaleurs de formation pour le \(CO_2\) et l'\(H_2O\) sont également nécessaires \cite{notice}. La valeur théorique obtenue pour le pouvoir calorifique de l'éthanol est donc \(H_{ethanol} = 26.8\) \si{\mega\joule\per\kilo\gram}.


\paragraph*{Analyse expérimentale de la combustion de l'éthanol}





\paragraph*{Pouvoir calorifique théorique du butane-propane}
A l'aide des \autoref{eq:chaleur_réaction}, \autoref{eq:butane_combustion} et \autoref{eq:propane_combustion} il est possible de calculer le pouvoir calorifique du mélange butane-propane, c'est-à-dire la chaleur que sa combustion libère. Il s'agit d'abord de calculer le pouvoir calorifique de ses deux composants. Pour le butane, sa chaleur de formation est de \(\Delta H_{g} = (-125.6 \pm 0.7)\) \si{\kilo\joule\per\mol} et sa masse molaire est de \(\rho = (58.12 \pm 0.01)\) \si{\gram\per\mol} \cite{butane-values}. Pour le propane, celles-ci sont \(\Delta H_{g} = (-104.7 \pm 0.5)\) \si{\kilo\joule\per\mol} et \(\rho = (44.10 \pm 0.01)\) \si{\gram\per\mol} \cite{propane-values}. A l'aide des valeurs des chaleurs de formation du \ce{CO2} et de l'\ce{H2O} \cite{notice} il est possible d'obtenir le pouvoir calorifique du butane \(H_{butane} = 45.7\) \si{\mega\joule\per\kilo\gram} et celui du propane \(H_{propane} = 46.3\) \si{\mega\joule\per\kilo\gram}.

Afin d'obtenir le pouvoir calorifique théorique du mélange butane-propane il est nécessaire de connaître les proportions de chaque composant. Dans le mélange utilisé le butane composait 80\% du mélange et le propane 20\%. Ainsi une






\paragraph*{Analyse expérimentale de la combustion du butane-propane}


