\section{Résultats}

\paragraph*{Pouvoir calorifique théorique de l'éthanol}
Conformément aux \autoref{eq:chaleur_réaction} et TODO EQUATION ETHAN il est possible de calculer le pouvoir calorifique de l'éthanol, c'est-à-dire la chaleur que sa combustion libère. Pour ce faire il est nécessaire de connaître sa chaleur de formation \(\Delta H_{l} = (-276 \pm 2)\) \si{\kilo\joule\per\mol} et sa masse molaire \(\rho = 46.07 \pm 0.01\) \si{\gram\per\mol} \cite{ethanol-values}. Les valeurs des chaleurs de formation pour le \(CO_2\) et l'\(H_2O\) sont également nécessaires \cite{notice}. La valeur théorique obtenue pour le pouvoir calorifique de l'éthanol est donc \(H_{ethanol} = 26.8\) \si{\mega\joule\per\kilo\gram}.


