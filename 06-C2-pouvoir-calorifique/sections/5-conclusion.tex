\section{Conclusion}

Cette expérience a permis de déterminer le pouvoir calorifique de deux combustibles, l'éthanol et un mélange butane-propane. Les résultats montrent que le mélange butane-propane possède un pouvoir calorifique presque 2 fois plus large que le mélange éthanol-méthanol ((\(41 \pm 1\)) contre (\(24 \pm 2\)) \si{\mega\joule\per\kilo\gram} respectivement). L'écart à la valeur théorique d'environ 10\% met en évidence les pertes lors de l'expérience. Cependant, la combustion de mélange éthanol-méthanol produit moins de \ce{CO2} que le mélange butane-propane: il faut donc trouver un équilibre entre fournir un chaleur plus forte et émettre plus de \ce{CO2}, un gaz à effet de serre.