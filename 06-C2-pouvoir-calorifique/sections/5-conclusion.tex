\section{Conclusion}

Cette expérience a permis de déterminer le pouvoir calorifique de deux combustibles, un mélange d'éthanol-méthanol et un mélange butane-propane. Les résultats montrent que le mélange butane-propane possède un pouvoir calorifique presque 2 fois plus large que le mélange éthanol-méthanol ((\(41 \pm 1\)) contre (\(24 \pm 2\)) \si{\mega\joule\per\kilo\gram} respectivement). L'écart à la valeur théorique d'environ 10\% met en évidence les pertes lors de l'expérience. La quantité de \ce{CO2} émise par la combustion permet aussi d'avoir une autre donnée permettant d'effectuer un choix entre plusieurs combustibles. Pour chauffer des habitations par exemple, il pourrait être plus avantageux d'utiliser un combustible avec un grand pouvoir calorifique et des grandes émissions plutot qu'un combustible moins puissant mais emettant moins de gaz à effet de serre.