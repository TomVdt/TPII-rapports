\section{Conclusion}

Cette expérience a permis de déterminer le pouvoir calorifique de deux combustibles, un mélange d'éthanol-méthanol et un mélange de butane-propane. Les résultats montrent que le mélange butane-propane possède un pouvoir calorifique presque 2 fois plus large que le mélange éthanol-méthanol ((\(41 \pm 1\)) contre (\(24 \pm 2\)) \si{\mega\joule\per\kilo\gram} respectivement). L'écart à la valeur théorique d'environ 10\% met en évidence les pertes lors de l'expérience. La quantité de \ce{CO2} émise par la combustion permet aussi d'avoir une autre donnée permettant d'effectuer un choix plus écologiques entre plusieurs combustibles. Dans le cadre du chauffage d'habitation par exemple, un choix doit être effectué entre les différents combustibles disponibles et des critères comme le pouvoir calorifique et l'émission de \ce{CO2} par masse brûlée peuvent être décisifs.