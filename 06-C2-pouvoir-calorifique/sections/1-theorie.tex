\section{Théorie}


\paragraph*{Réactions chimiques et pouvoir calorifique}
Il est possible de calculer théoriquement l'énergie libérée par une réaction chimique sous forme de chaleur à partir des chaleurs de formation des réactifs et des produits. Il faut pour cela étudier la réaction chimique sous la forme \mbox{\(\sum_{A} \nu_A X_A \to \sum_{B} \nu_B X_B\)} avec les réactifs \(X_A\), les produits \(X_B\) et les coefficients stoechiométriques \(\nu_i\). Pour cette réaction la chaleur de la réaction par moles d'avancement vaut:
\begin{equation}
    H = \sum_{A} \nu_A \Delta H_A - \sum_{B} \nu_B \Delta H_B
    \label{eq:chaleur_réaction}
\end{equation}
Avec les \(\Delta H_A\) et \(\Delta H_B\) les chaleurs de formation par moles des réactifs et des produits respectivement \cite{cours-thermo}. Pour un combustible brûlant avec uniquement de l'\(O_2\), ayant une chaleur de formation nulle \cite{notice}, la chaleur de la réaction donne directement la chaleur libérée par le combustible par \(\nu_{combustible}\) moles consommées. Pour obtenir le pouvoir calorifique du combustible il faut donc également connaître sa masse molaire afin de calculer la chaleur libérée par unités de masse consommées.

\paragraph*{Réactions de combustion}
Les combustibles étudiés dans cette expérience possèdent tous une équation chimique de combustion similaire. Ils consomment le combustible et de l'\ce{O2} pour produire du \ce{CO2} et de l'\ce{H2O}. Les chaleurs de formation de ces deux produits étant élevées et négatives cela permet un fort dégagement de chaleur au cours de la réaction \cite{notice}. Il est nécessaire cependant de connaitre ces équations de combustion pour les étudier, pour l'éthanol \ce{C2H5OH} sa combustion sous sa forme liquide donne:
\begin{equation}
    \ce{ C_2H_5OH_{(l)} + 3O_2_{(g)} -> 2CO_2_{(g)} + 3H_2O_{(g)} }
    \label{eq:ethanol_combustion}
\end{equation}
Pour le méthanol \ce{CH3OH} sous forme liquide:
\begin{equation}
    \ce{ 2CH3OH_{(l)} + 3O_2_{(g)} -> 2CO_2_{(g)} + 4H_2O_{(g)} }
    \label{eq:methanol_combustion}
\end{equation}
Pour le butane \ce{C4H10} sous forme gazeuse l'équation obtenue est:
\begin{equation}
    \ce{ 2C_4H_{10}_{(g)} + 13O_2_{(g)} -> 8CO_2_{(g)} + 10H_2O_{(g)} }
    \label{eq:butane_combustion}
\end{equation}
Et finalement pour le propane \ce{C3H8} gazeux:
\begin{equation}
    \ce{ C_3H_8_{(g)} + 5O_2_{(g)} -> 3CO_2_{(g)} + 4H_2O_{(g)} }
    \label{eq:propane_combustion}
\end{equation}
Ces 3 équations sont celles de la combustion complète en présence d'un apport d'oxygène suffisant pour remplir les conditions imposées par les coefficients stoechiométriques.

