\section{Théorie}


\paragraph*{Réactions et pouvoir calorifique}
Il est possible de calculer théoriquement l'énergie libérée par une réaction chimique sous forme de chaleur à partir des chaleurs de formation des réactifs et des produits. Il est possible d'étudier une réaction chimique sous la forme \(\sum_{A} \nu_A X_A \to \sum_{B} \nu_B X_B\) avec les réactifs \(X_A\), les produits \(X_B\) et les coefficients stoechimoétriques \(\nu\). Pour cette réaction la chaleur de la réaction par moles vaut:
\begin{equation}
    \mathcal{Q} = \sum_{A} \nu_A \Delta H_A - \sum_{B} \nu_B \Delta H_B
    \label{eq:chaleur_réaction}
\end{equation}
Avec les \(\Delta H_A\) et \(\Delta H_B\) les chaleurs de formation par moles des réactifs et des produits respectivement. Pour un combustible brûlant avec de l'\(O_2\), ayant chaleur de formation nulle \cite{notice}, la chaleur de la réaction donne directement la chaleur libérée par le combustible. Pour obtenir le pouvoir calorifique du combustible il faut donc également connaitre sa masse molaire.


\paragraph*{Pouvoir calorifique expérimental}
