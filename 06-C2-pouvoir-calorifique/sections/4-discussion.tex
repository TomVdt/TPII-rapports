\section{Discussion}



\paragraph*{Sources d'erreurs}
Le système expérimental utilisé possédait des éléments moins fiables que d'autres. Les mesures de la température où de la masse d'eau utilisée semble plutôt fiable. Cependant les faibles quantités de gazs brûlés rendent l'erreur sur leur mesure plus significative. La présence d'eau de condensation peut également être source d'erreur. En effet, sa mesure était imprécise car très basse mais dans l'expérience avec le gaz de camping la quantité était tout de même suffisante pour induire une erreur importante comme visible à l'aide des barres d'erreurs. Un dernier point est que l'erreur pour l'éthanol


\paragraph*{Erreur systématique}
Il est important de noter dans le cadre de cette expérience une importante source d'erreurs systématiques. En effet, lors de la réalisation de l'expérience, la flamme du combustible était simplement placée sous le calorimètre. Elle était donc en contact direct avec l'air extérieur et sans aucun dispositif d'isolation thermique à ce niveau là. Il est donc logique de supposer qu'une part significative de la chaleur produite par la combustion ait été évacuée vers l'extérieur sans chauffer l'intérieur du calorimètre. Cette perte de chaleur est probablement la cause de l'obtention de manière consistante de valeurs expérimentales inférieures à la valeur théorique. Ainsi s'il avait été possible de rediriger ou prendre en compte cette perte de chaleur, minoritaire mais significative, la valeur expérimentale aurait probablement été plus proche de celle attendue. Ce biais dans la mesure explique que même avec la marge d'erreur les valeurs expérimentales ne soient pas compatibles avec la valeur théorique.