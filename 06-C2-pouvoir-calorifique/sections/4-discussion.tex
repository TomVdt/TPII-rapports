\section{Discussion}

\paragraph*{Usage des combustibles}
Les résultats obtenus, théoriques comme expérimentaux, montrent comme attendu un très fort pouvoir calorifique de ces combustibles de l'ordre de \(10^2\) \si{\mega\joule\per\kilo\gram}. Cela a pu s'observer également expérimentalement par la possibilité de prendre de nombreuses mesures sans jamais avoir besoin de recharger ou changer le combustible tout en le faisant brûler continuellement. Les possibilités d'application de ces mélanges sont donc diverses grâce à la grande quantité d'énergie disponible pour une faible difficulté de transport.

\paragraph*{Cohérence des écart-types}
Les deux \autoref{fig:H_etanol} et \autoref{fig:H_gaz_camping} sont un bon exemple d'une bonne cohérence dans les résultats de l'analyse de données. En effet, il est possible d'observer qu'excepté deux valeurs singulières pour l'éthanol-méthanol l'ensemble des points obtenus sont compatibles, avec leur barre d'erreur, avec l'écart-type sur la moyenne. Cela montre que les mesures effectuées n'ont pas présenté trop d'anomalies et donc que le montage utilisé est bien stable. Il reste cependant certaines erreurs importantes qui sont notamment la cause de l'incompatibilité de la valeur théorique avec la valeur expérimentale pour le mélange butane-propane même en prenant les écart-types obtenus en compte. 


\paragraph*{Erreur systématique}
Il est important de noter dans le cadre de cette expérience une importante source d'erreurs systématiques. En effet, lors de la réalisation de l'expérience, la flamme du combustible était simplement placée sous le calorimètre. Elle était donc en contact direct avec l'air extérieur et sans aucun dispositif d'isolation thermique à ce niveau là. Il est donc logique de supposer qu'une part significative de la chaleur produite par la combustion ait été évacuée vers l'extérieur sans chauffer l'intérieur du calorimètre. Cette perte de chaleur est probablement la cause de l'obtention de manière consistante de valeurs expérimentales inférieures à la valeur théorique. Ainsi s'il avait été possible de rediriger ou prendre en compte cette perte de chaleur, minoritaire mais significative, la moyenne expérimentale aurait probablement été plus proche de celle attendue. Il est d'ailleurs intéressant de relever que pour la combustion du mélange éthanol-méthanol le brûleur avait pu être placé plus proche du dessous du calorimètre ce qui a été empêché pour le brûleur à butane-propane, correspondant ainsi au plus gros écart avec la théorie pour cette deuxième expérience.


\paragraph*{Sources d'erreurs}
Le système expérimental utilisé possédait des éléments moins fiables que d'autres. Les mesures de la température ou de la masse d'eau utilisées semblent plutôt fiables. Cependant les faibles quantités de gazs brûlés rendent l'erreur sur leur mesure plus significative. La présence d'eau de condensation peut également être source d'erreur. En effet, sa mesure était imprécise car très basse mais dans l'expérience avec le gaz de camping la quantité était tout de même suffisante pour induire une erreur importante comme visible à l'aide des grandes barres d'erreurs en \autoref{fig:H_gaz_camping}.
