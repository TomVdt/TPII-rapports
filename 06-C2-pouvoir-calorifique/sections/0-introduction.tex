\section{Introduction}

Dans le contexte de la crise climatique et énergetique, les moyens de chauffage restent une question majeure. En 2022, 56.8\% des habitations en Suisse étaient chauffées aux énergies fossiles, dont 39.3\% au mazout et 17.5\% au gaz \cite{chauffage}. Ces méthodes de chauffage fonctionnent sur le principe de combustion de liquide ou gaz afin de produire de la chaleur, permettant de réchauffer de l'eau, qui circule dans le circuit de chauffage par exemple. En plus de l'isolation des batiments, la qualité et la densité énergétique du combustible sont clés pour limiter l'impact du chauffage sur le climat. Il est estimé qu'environ 25\% de l'énergie produite dans le monde est utilisée pour réchauffer et refroidir les habitations et espaces commerciaux \cite{energie-chauffage}. Un enjeu essentiel est donc de connaître l'efficacité des différents combustibles.

Lors de cette expérience, le pouvoir calorifique ainsi que les émissions en \ce{CO2} de deux combustibles liquides: un gaz de camping composé à 80\% de butane et 20\% de propane, et un mélange éthanol-méthanol à 88\%-12\%, seront déterminés. De plus, la valeur théorique de ces combustibles sera calculée et comparée à la valeur expérimentale obtenue afin d'étudier la fiabilité du montage utilisé.