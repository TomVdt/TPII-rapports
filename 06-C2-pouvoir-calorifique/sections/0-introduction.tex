\section{Introduction}

\textcolor{red}{LET ME COOK \ce{C10H15N}}

Dans le contexte de la crise climatique et énergetique, les moyens de chauffage reste une question majeure. En 2022, 56.8\% des habitations étaients chauffés aux énergies fossiles, dont 39.3\% au mazout et 17.5\% au gaz \cite{chauffage}. Cette méthode de chauffage fonctionne sur le principe de combustion du liquide ou gaz afin de produire de la chaleur, permettant de réchauffer de l'eau, pouvant servir au chauffage par exemple. En plus de l'isolation des batiments, la qualité et la densité en chaleur du combustible sont clés pour limiter l'impact du chauffage sur le climat. Il est estimé qu'environ 25\% de l'énergie produite dans le monde est utilisée pour réchauffer et refroidir les habitations et espaces commerciaux \cite{energie-chauffage}. Un enjeu essentiel est donc de connaitre l'efficacité des combustibles.

Lors de cette expérience, le pouvoir calorifique ainsi que les émissions en \ce{CO2} de deux combustibles liquides: un gaz de camping composé à 80\% de butane et 20\% de propane, et un mélange éthanol-méthanol.