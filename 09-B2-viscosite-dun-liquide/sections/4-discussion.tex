\section{Discussion}

\paragraph{Viscosité} Les résutats obtenus montrent une différence entre la viscosité \(\eta\) obtenue pour une température croissante et pour une température décroissante, différence plus importante pour les températures plus élevées: la viscosité est plus élevée pour la température ascendante que descendante. Ces différences peuvent s'expliquer par plusieurs facteurs. La température de l'huile n'était pas forcemment homogène, et changeait pendant la chute de la bille. De plus, en raison de la manipulation pour récupérer la bille du fond du tube, l'huile n'était pas forcemment au repos. La mesure de la vitesse était aussi très imprecise en raison de la rapidité de la chute de la bille dans certains cas (moins de 4 secondes). Un dispositif avec des capteurs ou un enregistrement video permettrait d'augmenter grandement la précision de la mesure de l'écart de temps entre les marqueurs, et donc la vitesse. Face aux autres potentielles sources d'erreurs de mesure, celle-ci semble être la plus importante, vu que le chauffage et refroidissement de l'huile s'est déroulé sur une durée d'environ 1 heure alors que les chutes étaient de l'ordre de quelques secondes.

La viscosité de l'huile 1 était bien plus faible que celle de l'huile 2 à températures égales. Cependant pour de basses températures (\(T < 300\) \si{\kelvin}), l'huile 1 avait la même viscosité que l'huile 2 pour de hautes températures (\(T > 330\) \si{\kelvin}). Il est donc important de considerer la température d'utilisation d'une huile dans des applications.

\paragraph{Densité} Afin d'effectuer les calculs, la densité des huile a été supposée constante. La valeur de référence donnée était pour une température de 20 \si{\celsius}. Cependant, il était possible de voir pendant la manipulation la dillatation des huiles, puisque leur niveau montait. Il faudrait donc pouvoir mesurer la densité des huiles pour chaque température afin d'obtenir des résultats plus précis.



\paragraph{Approximation de Stokes} L'étude du nombre de Reynolds effectuée dans les \autoref{fig:huile1_Re} et \autoref{fig:huile2_Re} montre bien que l'approximation faite pour utiliser la formule de Stokes n'a pas du tout la bonne plage de validité. Ainsi l'\autoref{eq:trainee_stokes} utilisée dans toutes les études n'est pas forcément rtès exacte et cela aurait pu mener à des erreurs importantes. Ceci peut probablement au moins partiellement expliquer les écarts relatifs allant jusqu'à 30\% dans le \autoref{tab:val_ref}. Il est tout de même intéressant de noter que bien que les mesures ne soient pas très exactes elles sont tout de même plutôt fiables malgré cette approximation non fondée. Cela montre bien que le comportement général reste le même pour des valeurs de $\mathrm{Re} \leq 20$ bien qu'il ne soit plus recommandé d'utiliser l'approximation de Stokes pour obtenir des valeurs numériques précises.


\paragraph{Vitesse maximale} La comparaison a été faite entre le temps mis par la bille pour arriver à la position du début de la mesure de vitesse et la constante de temps de sa convergence vers sa vitesse maximale. Les valeurs étant très élevées ($t_1/\tau \geq 50$ pour l'huile 1 et $t_1/\tau \geq 200$ pour l'huile 2) cela indique que la vitesse maximale était bien atteinte avec dans le pire de tous les cas étudiés un écart de $10^{-25} \%$ ce qui est bien sûr hautement négligeable. Les sources d'erreurs de cette expérience ne viennent donc pas de cette supposition.