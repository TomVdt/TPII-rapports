\section{Discussion}

\paragraph{Viscosité} Les résutats obtenus montrent une différence entre la viscosité \(\eta\) obtenue pour une température croissante et pour une température décroissante, différence plus importante pour les températures plus élevées: la viscosité est plus élevée pour la température ascendante que descendante. Ces différences peuvent s'expliquer par plusieurs facteurs. La température de l'huile n'était pas forcemment homogène, et changeait pendant la chute de la bille. De plus, en raison de la manipulation pour récupérer la bille du fond du tube, l'huile n'était pas forcemment au repos. La mesure de la vitesse était aussi très imprecise en raison de la rapidité de la chute de la bille dans certains cas (moins de 4 secondes). Un dispositif avec des capteurs ou un enregistrement video permettrait d'augmenter grandement la précision de la mesure de l'écart de temps entre les marqueurs, et donc la vitesse. Face aux autres potentielles sources d'erreurs, celle-ci semble être la plus importante, vu que le chauffage et refroidissement de l'huile s'est déroulé sur une durée d'environ 1 heure alors que les chutes étaient de l'ordre de quelques secondes.

La viscosité de l'huile 1 était bien plus faible que l'huile 2 à la même température, mais à faible température (\(T < 300\) \si{\kelvin}), l'huile 1 avait la même viscosité que l'huile 2 à haute température (\(T > 330\) \si{\kelvin}). Il est donc important de considerer la température d'utilisation d'une huile dans des applications.

\paragraph{Densité} Afin d'effectuer les calculs, la densité des huile a été supposée constante. La valeur de référence donnée était pour une température de 20 \si{\celsius}. Cependant, il était possible de voir pendant la manipulation la dillatation des huiles, puisque leur niveau montait. Il faudrait donc pouvoir mesurer la densité des huiles pour chaque température afin d'obtenir des résultats plus précis.
