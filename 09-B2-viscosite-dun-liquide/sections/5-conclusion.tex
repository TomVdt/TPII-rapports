\section{Conclusion}

Lors de cette expérience, la viscosité de deux huiles, HF 16-681 et HF 16-685, à différentes températures, ont été étudiées. Il a été observé qu'une température ascendante ou descendante influençait les résultats. L'énergie d'activation, ainsi que le nombre de Reynolds, ont été déterminées pour chaque huile. Une étude de la constante de temps a aussi été réalisée. Il est donc possible de voir que la température a un grand effet sur la viscosité des huiles étudiées. Lors d'applications dans l'industrie, il est donc essentiel de connaitre la température d'opération de machines necessitant des huiles pour la lubrification, ainsi que le réchauffement subit par l'huile à cause des frottements \cite{lubrifiant_uwu}.
