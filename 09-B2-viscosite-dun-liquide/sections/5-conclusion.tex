\section{Conclusion}

Lors de cette expérience, les viscosités de deux huiles, HF 16-681 et HF 16-685, ont été étudiées à différentes températures. Il a été observé qu'une température ascendante ou descendante influençait les résultats. L'énergie d'activation a été déterminée pour chaque huile. Une étude du nombre de Reynolds ainsi que de la constante de temps a aussi été réalisée pour chaque chute de la bille afin d'étudier les erreurs mesurées. Il est donc possible de voir que la température a un grand effet sur la viscosité des huiles étudiées. Lors d'applications dans l'industrie, il est donc essentiel de connaitre la température d'opération de machines utilisant des huiles pour la lubrification, ainsi que le réchauffement subit par l'huile à cause des frottements \cite{lubrifiant_uwu}.
