\section{Introduction}
Les fluides ça fait pschit pschit

Les liquides sont utilisés dans de très nombreux domaines pour diverses applications. Dans la grande majorité des cas, ils sont mis en mouvement afin de profiter de leur propriétés mécaniques intéressantes. Cette mise en mouvement est cependant entravée par une friction provenant de leur viscosité. Cela peut rapidement avoir des impacts importants sur leur utilisation comme par exemple dans le cas des huiles de moteur qui doivent avoir exactement la viscosité souhaitée sous peine d'augmenter l'usure du moteur \cite{roger_le_mecano}. De plus, cette viscosité peut varier en fonction de la température ce qui pourrait aussi mener à d'autres effets pas forcément voulus. Ainsi il devient important de mesurer précisément la viscosité des huiles et autres liquides dans une large gamme de température.

Pour ce faire, l'expérience présentée utilise un viscosimètre à bille. Lors de cette expérience les viscosités de deux huiles seront mesurées en fonction de la température afin d'établir le lien entre ces deux grandeurs. L'exactitude des mesures effectuées sera ensuite étudiée en analysant les plages de validité des approximations utilisées et en les comparant aux nombre de Reynolds et temps obtenus.