\section{Discussion}

\paragraph{Angle d'incidence}
\(I_\gamma\) varie en fonction de l'angle d'incidence de manière sinusoïdale et permet d'analyser un facteur important de l'installation de panneau solaire, l'orientation par rapport au soleil. Cela indique clairement que de faibles angles n'auront qu'un impact modéré sur l'efficacité mais qu'il faut éviter autant que possible les angles trop élevés au risque d'avoir un courant quasi nul.

\paragraph{Données du fabricant}
Le fabricant de la cellule monocristalline utilisée dans cette expérience indique un rendement de 10\% sous éclairage de \(1000\) \unit{\watt \per \square \meter}. Bien que les valeurs du rendement établies lors de cette expérience, 4.5\% à 4.7\%, sont bien plus faibles que celles indiquées les conditions d'éclairage et de mesure ne sont pas les mêmes. Les limites du laboratoire peuvent être responsables de cette écart, comme la puissance lumineuse maximale atteinte (de seulement \(283\) \unit{\watt \per \square \meter}) ou l'imperfection de la mesure dues aux appareils moins précis. Ainsi il n'est pas possible de déterminer avec exactitude la véracité de cette valeur à l'aide du montage réalisé.

\paragraph{Erreurs}
Il semble judicieux de mettre en évidence les plusieurs points de défaillance de cette expérience. Tout d'abord certains appareils utilisés lors de la mesure de la puissance ne permettent pas une grande fiabilité, tels qu'un simple fil de métal permettant une résistance proportionelle à la longueur utilisée. De plus il est important d'ajouter que certaines erreurs systématiques viennent s'ajouter et notamment la résistance interne de l'ampèremètre qui ne permet donc pas de mesure réelle du courant de court-circuit. Une autre erreur systématique vient de la présence de murs blancs et d'une fenêtre non-couverte dans la salle venant rajouter une constante à l'intensité lumineuse fournie. Celle-ci peut être négligée dans la grande majorité des expériences grâce à la grande puissance de la lampe à mercure, cependant lorsque l'angle d'incidence devient important cette intensité devient significative et empêche notamment la sinusoïde d'atteindre \(0\) \unit{\ampere}.