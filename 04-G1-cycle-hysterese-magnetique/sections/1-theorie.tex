\section{Théorie}

Lorsqu'un courant intense traverse un transformateur composé de nombreuses spires il se crée une induction magnétique \(\vec{B}\) dont l'intensité s'exprime par l'équation:
\begin{equation}
    B = \mu_0 I \frac{N}{L} = \mu_0 \frac{V_i}{R} \frac{N}{L}
    \label{eq:B_I}
\end{equation}
Dans laquelle \(I\) est le courant qui traverse le transformateur, \(N\) le nombre de spires du primaire du transformateur et \(L\) sa longueur \cite{assistant}. La deuxième équation est obtenue par la loi d'Ohm \(V_i = RI\) avec \(R\) la résistance placée à l'entrée du transformateur et \(V_i\) la tension mesurée à ses bornes.

De plus dans un matériau donné la relation entre l'induction magnétique \(B\) et le champ magnétique \(H\) est donnée par:
\begin{equation}
    B = \mu_0 (\chi_m + 1) H = \mu_0 \mu_r H
    \label{eq:B_H}
\end{equation}
Avec \(\chi_m\) la susceptibilité magnétique du matériau et \(\mu_r = \chi_m + 1\) sa perméabilité relative \cite{notice}. La perméabilité absolue du vide est exactement définie par \(\mu_0 = 4\pi \times 10^{-7}\) \si{\volt\second \per\ampere\per\meter}.

De la même manière que le primaire du transformateur crée une induction magnétique, cette induction magnétique produit également un courant dans le secondaire proportionnel par la loi d'Ohm à la tension \(V_f\) mesurée à sa sortie. Ainsi \(V_i\) et \(V_f\) sont proportionnelles à \(B\) donc il est possible d'écrire une relation de proportionnalité entre elles en définissant \(\alpha\) telle que:
\begin{equation}
    V_f = \beta V_i = \alpha \mu_r V_i
    \label{eq:Vi-Vf}
\end{equation}

En combinant les équations \autoref{eq:B_I} et \autoref{eq:B_H} il vient:
\begin{equation}
    H = \frac{N}{\mu_r L} \frac{V_i}{R}
    \label{eq:calibr_H}
\end{equation}
Et avec la relation entre \(V_i\) et \(V_f\) donné par l'\autoref{eq:Vi-Vf} cela donne également:
\begin{equation}
    B = \mu_0 \frac{N}{L} \frac{1}{\alpha \mu_r} \frac{V_f}{R}
    \label{eq:calibr_B}
\end{equation}

De plus l'\autoref{eq:B_H} permet de conclure que \(\mu_r = 1\) et donc \(V_f = \alpha V_i\) pour le vide. Ainsi il faut prendre la régression linéaire des mesures sans échantillons pour le système étudié pour connaître \(\alpha\). Il est ensuite possible de trouver la valeur de \(\mu_r\) en connaissant les coefficients de proportionnalité seulement des tensions qui sont plus simples à mesurer que des inductions et des champs magnétiques:
\begin{equation} 
    \mu_r = \frac{\beta}{\alpha}
    \label{eq:mu_r}
\end{equation} 


Maintenant, il est possible de classifier les matériaux par leur comportement magnétique. Les matériaux diamagnétiques possèdent une perméabilité au magnétisme inférieure à celle du vide car pour ces matériaux \((\chi_m < 0 \Rightarrow \mu_r < 1\)). Ceux paramagnétiques possèdent des moments magnétiques internes les rendant plus susceptible à l'induction magnétique avec \(chi_m > 0\). Finalement les matériaux ferromagnétiques présentent des interactions entre ces moments magnétiques internes qui restent même après la disparition du champ magnétique créant ainsi un phénomène d'hystérèse.

\begin{wrapfigure}{R}{0.6\textwidth}
    \centering
    \includegraphics[width=0.7\linewidth]{figures/cycle_hysterese.png}
    \caption{Graphe d'un cycle d'hystérèse magnétique avec ses valeurs caractéristiques \cite{graph_cycle}}
    \label{fig:cycle}
\end{wrapfigure}
Les cycles d'hystérèse magnétique présentent deux couples de valeurs caractéristiques visibles dans la \autoref{fig:cycle}. Le premier est le champ magnétique \(H_S\) et l'induction \(B_S\) de saturation qui correspondent au niveau de magnétisation maximale du matériau considéré. Au-delà de ce champ magnétique la valeur de l'induction présente dans le matériau n'augmentera plus. Le deuxième couple de valeurs est l'induction rémanente \(B_r\) et le champ coercitif \(H_C\). Ils correspondent respectivement à l'induction encore présente au sein du matériau même après la mise à zéro du champ magnétique appliqué et au champ magnétique nécessaire pour venir compenser complètement cette induction rémanente.