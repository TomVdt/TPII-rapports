\section{Discussion}

\paragraph*{Étalonnage}
La précision de l'étalonnage du transformateur PHYWE à vide est possiblement peu précis. Sur la \autoref{fig:phywe_vide} il est possible de voir qu'il y a un léger décalage vers le haut des valeurs \(V_y\) mesurées, suite à la première partie du cycle \(0 \rightarrow V_+\). Il est probable que la source de ce décalage soit la dérive de l'intégrateur.

\paragraph*{Types de magnétisation}
Plusieurs types de magnétisation des matériaux ont été trouvés au cours de cette expérience. Tout d'abord les deux matériaux contenant de l'acier ont été déterminés comme fortement ferromagnétiques avec de hautes valeurs pour \(mu_r\) dans le transformateur cylindre. Il est donc possible de déduire que l'acier en général est un alliage avec un caractère ferromagnétique et une très bonne perméabilité au magnétisme. Les autres caractéristiques obtenues sont moins prononcées avec des \(chi_m\) plus proches de 0. 

Deux matériaux ont tout de même été identifés comme paramagnétiques cependant cette classification n'a pas pu être vérifiée sur un autre transformateur par manque d'échantillons adaptés. La présence de deux classifications contradictoire selon le transformateur indique que cette double vérification pourrait être utile. En effet l'aluminium et le cuivre ont tous deux donnés des résultats différents selon le transformateur utilisé indiquant que cette méthode de détermination n'est pas suffisament fiable quand \(\chi_m\) est proche de 0. Les valeurs obtenues sont tout de même proche de celles attendues, en effet la littérature indique que le cuivre est diamagnétique et l'aluminium paramagnétique avec des valeurs de \(\chi_m\) faibles \cite{classification_litt}. 

L'erreur sur la classification produite provient probablement d'une erreur systématique sur l'un des deux, ou les deux, transformateurs. En effet, les valeurs obtenues pour l'aluminium et le cuivre sont toutes les deux supérieures à 1 pour le PHYWE et inférieure à 1 pour le cylindrique. Il est donc plausible qu'une erreur sur la mesure à vide liée à la détermination de \(\mu_r\) ait faussé ces données suffisament pour que sur les faibles valeurs la différence soit significative. Une autre possibilité est que le montage utilisé avec des échantillons déjà utilisé précédemment et simplement posés à l'endroit de la mesure ne soit pas suffisament précis pour des valeurs aussi délicates que celles du cuivre et de l'aluminium.

\paragraph*{Calibration}
La calibration des axes a permis de mieux visualiser les valeurs physiques de ces cycles d'hystérèse afin d'avoir une compréhension plus intuitive que simplement sur la base de tensions mesurées. Il a été possible d'observer les carctéristiques de l'hystérèse d'un matériau ferromagnétique doux, le bloc PHYWE, et de constater que ses valeurs de champs coercitifs sont très faibles. C'est donc un matériau facile à démagnétiser. Les valeurs précises des \(H_C\), \(H_S\), \(B_r\) et \(B_S\) n'ont pas pu être déterminées car les très nombreux points disponibles permettent des régressions linéaires précises et des graphiques clairs mais compliquent l'extraction de valeurs. En effet il aurait été possible de choisir le point le plus proche de 0 \si{\milli\tesla} par exemple mais cela aurait probablement été source de grandes erreurs car il suffirait d'un point mal placé pour qu'il soit sélectionné à tort par le programme de traitement des données. De plus la courbe de première magnétisation qui débute en 0 garantirait presque l'existence de cette erreur. Les limitations de traitements de données contraignent donc ici à une lecture du graphique pour connaître l'ordre de grandeur sans plus de précision.

\paragraph*{Autres expériences}
La variation de l'épaisseur des échantillons montre que les cycles d'hystérèse des matériaux ferromagnétique dépendent de la géométrie de l'échantillons. La différence entre les échantillons rectangulaires pour le PHYWE et cylindriques pour le transformateur cylindrique ne permet pas de conclure sur l'effet de la géométrie, en vue des différences entre les deux transformateur. Afin de mieux mesurer l'effet de la géométrie du matériaux il serait bien de mesurer différentes formes d'échantillons sur le même transformateur, avec les mêmes paramètres.

Les courbes d'aimantation des matériaux paramagnétiques dépendent aussi de la température du matériau. Une expérience interessante pourrait être de mesurer la réaction des matériaux à différentes températures, pour mesurer les effets de la température sur les matériaux. Pour les matériaux ferromagnétique, il serait également interessant de déterminer la température de Curie, au delà de laquelle ces matériaux perdent leur magnétisme.
