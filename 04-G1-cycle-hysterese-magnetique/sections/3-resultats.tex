\section{Résultats}

\paragraph{Étalonnage des transformateurs}
Les cycles à vide des transformateurs permettent de trouver \(\alpha\) tel que \(V_f = \alpha V_i\). Une régression linéaire sur le cycle du transformateur PHYWE à la \autoref{fig:phywe_vide} permet d'obtenir \mbox{\(V_f = (0.069 \pm 0.003) V_i\)}. Une régression linéaire sur le cycle du transformateur cylindrique à la \autoref{fig:cylindre_vide} donne \mbox{\(V_f = (0.0191 \pm 0.0001) V_i\)}. Le transformateur cylindrique ne présente pas d'hystérèse à vide, alors que le transformateur PHYWE possède une légère hystérèse, mais l'écart entre les courbes en montée et en descente est très faible, et les courbes restent linéaires sur une majeure partie du graphe.

\begin{figure}[h]
    \centering
    \begin{subfigure}{0.5\linewidth}
        \centering
        \includegraphics[width=\linewidth]{figures/G1-phywe-vide.pdf}
        \caption{}
        \label{fig:phywe_vide}
    \end{subfigure}%
    \begin{subfigure}{0.5\linewidth}
        \centering
        \includegraphics[width=\linewidth]{figures/G1-cylindre-vide.pdf}
        \caption{}
        \label{fig:cylindre_vide}
    \end{subfigure}
    \caption{Cycles à vide sur les transformateur (a) PHYWE (b) cylindrique}
\end{figure}

\paragraph{Propriétés des matériaux}
Les différents matériaux mesurés ont présenté des comportements magnétiques différents. Leur classification a pu être déterminée en connaissant leur valeur de \(\mu_r\) qui a été obtenue par l'\autoref{eq:mu_r} à l'aide des régressions linéaires sur les transformateurs à vide et sur les parties linéaires des graphiques des différents matériaux. Pour les matériaux diamagnétiques et paramagnétiques présentant un comportement purement linéaire cela représente l'entièreté des points relevés. Pour ceux présentant un cycle d'hystérèse la partie linéaire aux extremités du graphe a été extraite. La classification obtenue avec les valeurs des \(\mu_r\) est présentée dans le \autoref{tab:mu_r}.

\begin{table}[h]
    \vspace{5pt}
    \centering
    \begin{adjustbox}{width=\textwidth}
        \begin{tabulary}{1.2\linewidth}{|c c c c c c c|}
            \toprule
            & Acier doux & Aluminium & Cuivre & Acier-Ag-Cr & Nickel-200 & Monel-400 \\
            \midrule
            \(\mu_r\) pour PHYWE & \(1.16 \pm 0.04\) & \(1.08 \pm 0.01\) & \(1.08 \pm 0.01\) & - & - & - \\
            \(\mu_r\) pour Cylindre & \(2.48 \pm 0.08\) & \(0.99 \pm 0.01\) & \(0.98 \pm 0.01\) & \(2.1 \pm 0.1\) & \(1.11 \pm 0.01\) & \(1.09 \pm 0.01\) \\
            Magnétisme & Ferro- & - & - & Ferro- & Para- & Para- \\
            \bottomrule
        \end{tabulary}
    \end{adjustbox}
    \caption{Valeurs de \(\mu_r\) pour différents échantillons dans chaque transformateur et leurs types de magnétisme (Ferro-, Para- et Dia- magnétisme)}
    \label{tab:mu_r}
\end{table}


\begin{minipage}{\linewidth}
    \begin{wrapfigure}{R}{0.5\linewidth}
        \includegraphics[width=\linewidth]{figures/G1-phywe-avec-bloc_chang.pdf}
        \caption{Cycle d'hystérèse du bloc PHYWE}
        \label{fig:calibr_phywe}
    \end{wrapfigure}

    \paragraph{Callibration des axes}
    A partir des équations \autoref{eq:calibr_H} et \autoref{eq:calibr_B} il est possible de calculer les constantes pour pouvoir changer de variables et obtenir une induction magnétique en fonction d'un champ. En prenant pour le transformateur PHYWE \hbox{\(L = 6.4 \pm 0.1\) \si{\centi \meter}} et \(N = 600\) le nombre de spire du primaire, les coefficients obtenus lorsque le bloc PHYWE est utilisé sont: \hbox{\(H = (4.6\pm0.1)\times10^3 V_i\)} et \hbox{\(B = (8.3\pm0.2)\times10^{-2} V_f\)}. La figure avec les axes calibrés est visible en \autoref{fig:calibr_phywe}.
    Ces nouveaux graphiques permettent de mieux visualiser le système physique et les caractéristiques principales de son hystérèse à savoir les valeurs des \(H_c\), \(H_S\), \(B_r\) et \(B_S\). Pour le bloc PHYWE la valeur du champ coercitif après saturation est donc très proche de 0 \si{\kilo\ampere\per\meter}. Les valeurs de saturation  du champ magnétique et de l'induction respectivement peuvent être lues aux alentour de 10 \si{\kilo\ampere\per\meter} et 200 \si{\milli\tesla}.

    TODO: donner les caractéristique du cycle d'hystérèse
    TODO: citer l'annexe pour dire "il y a d'autres cycles possibles"
\end{minipage}

\paragraph{Échantillons empilés}
La \autoref{fig:combo} est obtenue en empillant deux échantillons du même matériau. Les épaisseurs des échantillons sont dans le \autoref{tab:thiccness} en \autoref{sec:resultats_bonus}. Pour les matériaux ferromagnétiques comme l'acier à la \autoref{fig:acier_combo}, la tension de sortie est presque doublée entre une simple et une double couche. Leur réponse au champ magnétique varie donc en fonction de l'épaisseur du matériau.Cependant, les autres échantillons, comme l'aluminium à la \autoref{fig:alu_combo} ne réagissent pas différement si l'épaisseur est changée.

\begin{figure}[h]
    \centering
    \begin{subfigure}{0.5\linewidth}
        \centering
        \includegraphics[width=\linewidth]{figures/ac_doux_simple_vs_combo.pdf}
        \caption{}
        \label{fig:acier_combo}
    \end{subfigure}%
    \begin{subfigure}{0.5\linewidth}
        \centering
        \includegraphics[width=\linewidth]{figures/alu_simple_vs_combo.pdf}
        \caption{}
        \label{fig:alu_combo}
    \end{subfigure}
    \caption{Comportement d'une simple et double couche pour des échantillons en (a) acier doux (b) aluminium}
    \label{fig:combo}
\end{figure}

\paragraph{Plaques amagnétiques}
L'insertion de plaques amagnetique entre le transformateur et le bloc PHYWE change aussi la réponse du bloc, comme le montre la \autoref{fig:amagnetique}. Un nombre plus grand, correspondant à une plus grande épaisseur, de plaques amagnétiques diminue et même élimine le cycle d'hystérèse du bloc, qui répond alors linéairement au changement de champ magnétique.

\paragraph{Orientation du bloc PHYWE}
Le bloc PHYWE est composé de plusieurs couches de fer \cite{bloc_phywe}. La \autoref{fig:orientation_bloc} donne la réponse du bloc PHYWE en fonction de l'orientation des lamelles. Lorsque les lamelles sont orientées dans un sens différent que prévu, ici verticallement, le champ magnétique est moins fort que pour l'orientation prévue, avec les lamelles horizontales.

\begin{figure}[h]
    \begin{minipage}{0.48\linewidth}
        \centering
        \includegraphics[width=\linewidth]{figures/separateurs_amagnetique.pdf}
        \caption{Réponse du bloc PHYWE après insertion de 1, 2 et 6 séparateurs amagnétiques}
        \label{fig:amagnetique}
    \end{minipage}
    \hfill
    \begin{minipage}{0.48\linewidth}
        \centering
        \includegraphics[width=\linewidth]{figures/vertical_vs_horizontal.pdf}
        \caption{Comportement du bloc PHYWE en fonction de l'orientation des lamelles}
        \label{fig:orientation_bloc}
    \end{minipage}
\end{figure}
    
