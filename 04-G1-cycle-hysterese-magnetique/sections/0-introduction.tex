\section{Introduction}

La production de champs magnétiques est au coeur de très nombreuses technologies. Elle est notamment au coeur du fonctionnement des accélérateurs de particules qui utilisent les champs magnétiques pour donner de l'énergie aux particules étudiées. Cependant la réponse des matériaux à la création et l'application d'un champ magnétique peut dépendre de nombreux facteurs. En particulier pour certains matériaux cela dépend de leur histoire de magnétisation tel les aimants du Large Hadron Collider du CERN qui nécessitent un entraînement avant leur mise en route à pleine puissance \cite{CERN_aimants}. Il est donc important de comprendre ces phénomènes qui sont caractérisés par les cycles d'hystérèse.

L'expérience présentée consiste à déterminer les caractéristiques magnétiques de différents matériaux. Pour ce faire la réponse de ces matériaux à différents champs magnétiques doit être mesurée et étudiée. Plusieurs transformateurs ont été utilisés pour générer ces champs magnétiques afin d'avoir des mesures plus variées.