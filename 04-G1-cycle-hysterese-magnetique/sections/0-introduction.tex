\section{Introduction}

La production de champs magnétiques est au coeur de très nombreuses technologies. Elle est notamment au coeur du fonctionnement des accélérateurs de particules qui utilisent les champs magnétiques pour donner de l'énergie aux particules étudiées. Cependant la réponse des matériaux à la création et l'application d'un champ magnétique peut dépendre de nombreux facteurs. En particulier pour certains matériaux cela dépend de leurs magnétisations passées tel les aimants du Large Hadron Collider du CERN qui nécessitent un entraînement avant leur mise en route à pleine puissance, sans quoi ce brusque changement pourrait endommager voir brûler ces aimants \cite{CERN_aimants}. Il est donc important de comprendre ces phénomènes qui peuvent être caractérisés par des cycles d'hystérèse.

L'expérience présentée consiste à déterminer les caractéristiques magnétiques de différents matériaux. Pour ce faire la réponse de ces matériaux à différents champs magnétiques doit être mesurée et étudiée. Afin d'avoir des mesures plus variées plusieurs transformateurs ont été utilisés pour générer ces champs magnétiques et étudier ces caractéristiques.