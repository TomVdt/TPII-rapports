\section{Calcul d'erreurs}

Les erreurs sur les mesures sont données dans le \autoref{tab:erreurs}.

\begin{table}[h]
    \centering
    \begin{tabulary}{\textwidth}{C C C}
        \toprule
        Variable & Erreur & Commentaire \\
        \midrule
        \(a\) [\si{\micro\meter}] & 0.01 & Estimation grossière basée sur l'epaisseur indiquée sur les échantillons \\
        \(B\) [\si{\milli\tesla}] & \(1\% + 3\textrm{ lsd}\) & Indiqué sur le manuel du teslamètre \\
        \(V\) [\si{\milli\volt}] & \(0.01\) & Estimation basée sur la variation des valeurs affichées \\
        \(I\) [\si{\milli\ampere}] & \(0.001\) & Estimation basée sur la variation des valeurs affichées \\
        \(I\) [\si{\ampere}] & \(0.01\) & Estimation basée sur la variation des valeurs affichées \\
        \bottomrule
    \end{tabulary}
    \caption{Erreurs estimées sur les mesures}
    \label{tab:erreurs}
\end{table}

\paragraph*{Regression linéaire}
Les erreurs sur les fits linéaires \(y = ax + b\) sur les mesures \((x_i, y_i) ; i = \{1, \hdots, n\}\) sont donnés par l'\autoref{eq:erreur:fit} \cite{erreursmesure}:

\begin{equation}
    \label{eq:erreur:fit}
    \begin{aligned}
        (\Delta a)^2 &= \frac{\sum_{i=1}^{n}(y_i - (a x_i + b))^2}{(n-2) \sum_{i=1}^{n}(x_i - \bar{x})^2}\\
        \Delta b &= \bar{x} \Delta a + a \Delta \bar{x}
    \end{aligned}
\end{equation}

En pratique, ces valeurs sont calculées par la bilbiothèque python \texttt{numpy}.

\paragraph*{Equation 1}
Blabla

Toutes ces erreurs sont calculées en pratique par la bibliothèque \texttt{uncertainties}.
