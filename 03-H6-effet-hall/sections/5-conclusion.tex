\section{Conclusion}

L'effet Hall a été observé sous plusieurs conditions différentes durant cette expérience. Dans un premier temps la présence d'une hystérèse dans la production du champ magnétique à été mise en évidence. Ensuite, la constante de Hall \(R_H\) et la densité de porteurs de charges \(N\) a été déterminée pour deux échantillon de semiconducteurs et trois échantillons de métal, à l'aide de deux méthodes différentes, en faisant varier le champ magnétique \(B\) et en faisant varier l'intensité du courant \(I\). Cela a permis de confirmer que les porteurs de charge dans ces échantillons étaient les électrons. Les tensions mesurées étant très faibles, de l'ordre du \si{\milli\volt}, des applications réelles de l'effet Hall nécessiteraient un traitement du signal obtenu et notamment une amplification.