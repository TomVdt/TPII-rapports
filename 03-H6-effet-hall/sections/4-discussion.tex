\section{Discussion}

\paragraph*{Calibration du teslamètre}
La \autoref{fig:callibration} met en évidence la non-linéarité du champ magnétique \(B\) en fonction du courant \(I\). Le domaine utile de \(B\) se situe donc dans la partie linéaire, pour \(I \in [-3, 3]\) \si{\ampere}. La présence d'une hystérèse indique aussi qu'il faut faire attention au sens du cycle lors des mesures, qui ont été faites dans le sens \(B_-\) à \(B_+\) pour cette expérience. L'incertitude sur \(B\) serait donc bien plus grande si cette convention n'avait pas été suivie, de l'ordre de 10 \si{\milli\tesla} d'après les regressions linéaires.

\paragraph*{Echantillons à 4 contacts}
La présence d'une tension résiduelle dans ces échantillons interfère avec les résultats. Elle est responsable notamment du décalage des régressions linéaires dans les \autoref{fig:inpV(I)} et \autoref{fig:inpV(B)}, qui devraient passer proche du 0. Les sources de cette tension résiduelle sont multiples. En effet des champs magnétiques sont créés par les installations éléctriques tout autour du montage, par les alimentation, les fils, les cables dans les murs ou encore le champ magnétique terrestre. Pour limiter ces effets secondaires, il serait possible d'isoler mieux les échantillon lors de ces mesures en utilisant par exemple une cage Faraday ou d'autres matériaux plus exotiques \cite{em_shielding}.\\
Les valeurs de la constante de Faraday \(R_H\) diffèrent un peu aussi selon la méthode utilisée. Il est possible que cela est causé par une imprécision sur la mesure du champ magnétique avec le teslamètre qui est très sensible à l'orientation de la sonde. Le champ magnétique maximal obtenu était de \((348 \pm 7)\) \si{\milli\tesla} pour l'échantillon de 1 \si{\micro\meter} et de \((383 \pm 7)\) \si{\milli\tesla} pour celui de 2 \si{\micro\meter}, alors que le courant d'alimentation des bobines était le même.


\paragraph*{Echantillons à 5 contacts}
Le potientomètre a été utile pour se débarasser de la tension résiduelle. Cela provient de la chute de tension venant de la loi d'Ohm: \(U = R I\). Ainsi en introduisant une résistance il est possible de trouver sa valeur pour supprimer complètement cette tension résiduelle. Il a été nécessaire de prendre des grandes résistances (de l'ordre du \si{\kilo \ohm}) car le courant étant très faible la chute de tension l'était également.

Les valeurs de N trouvés sont cohérentes et sont très importantes de l'odre de \(10^{25}\), \(10^{28}\) et \(10^{29}\). Il est cependant important de noter l'écart de 4 ordres de grandeur entre l'argent et le cuivre, et le bismuth. Le bisumuth ayant le moins de porteur de charges par unité de volume il sera donc moins propice pour observer l'effet Hall qui dépend du nombre de porteurs de charge déplacés par le champ magnétique. Cela correspond avec l'observation qu'il possède le plus grand coefficient de Hall \(R_H = (-3.62 \pm 0.07) \times 10^{-7}\) \si{\cubic \meter \per \coulomb} qu'il est possible d'assimiler à une forme de résistance à l'effet Hall par similaritude avec la loi d'Ohm: \(V_H = -R_H (\frac{IB}{a})\).

Les valeurs obtenues sont toutes du même ordre de grandeur que les valeurs tabulées dans la littérature \cite{notice} et semblent cohérentes. Elles ne sont tout de même pas toutes proches de ce qui était attendu ce qui peut être le résultat des sources d'imprécision importantes de ce montage. La mesure à l'aide du teslamètre notamment n'était que peu précise car très sensible au moindre décalage dans sa position. Les mesures obtenues ont tout de même été particulièrement propices à des régressions linéaires avec de très faibles erreurs sur les fits grâce à l'excellent alignement des mesures. Cette erreur systématique a donné des mesures avec de faibles incertitudes bien que significativement éloignées de celles attendues. Le montage a donc bien marché mais les mesures étaient polluées par l'erreur systématique due à la position du teslamètre.