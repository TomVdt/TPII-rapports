\section{Discussion}

\paragraph*{Calibration du teslamètre}
La \autoref{fig:callibration} met en évidence la non-linéarité du champ magnétique \(B\) en fonction du courant \(I\). Le domaine utile de \(B\) se situe donc dans la partie linéaire, pour \(I \in [-3, 3]\) \si{\ampere}. La présence d'une hystérèse indique aussi qu'il faut faire attention au sens du cycle lors des mesures, qui ont été faites dans le sens \(B_-\) à \(B_+\) pour cette expérience. L'incertitude sur \(B\) serait donc bien plus grande si cette convention n'avait pas été suivie, de l'ordre de 10 \si{\milli\tesla} d'après les regressions linéaires.

\paragraph*{Échantillons à 4 contacts}
La présence d'une tension résiduelle dans ces échantillons interfère avec les résultats. Elle est responsable notamment du décalage des régressions linéaires dans les \autoref{fig:inpV(I)} et \autoref{fig:inpV(B)}, qui devraient passer proche du 0. Les sources de cette tension résiduelle sont multiples. Premièrement, des champs magnétiques sont créés par les installations éléctriques tout autour du montage, par les alimentation, les fils, les cables dans les murs ou encore le champ magnétique terrestre. Pour limiter ces effets secondaires, il serait possible d'isoler mieux les échantillon lors de ces mesures en utilisant par exemple une cage Faraday ou d'autres matériaux plus exotiques \cite{em_shielding}.\\
Les valeurs de la constante de Faraday \(R_H\) diffèrent un peu aussi selon la méthode utilisée. Il est possible que cela est causé par une imprécision sur la mesure du champ magnétique avec le teslamètre qui est très sensible à l'orientation de la sonde. Le champ magnétique maximal obtenu était de \((348 \pm 7)\) \si{\milli\tesla} pour l'échantillon de 1 \si{\micro\meter} et de \((383 \pm 7)\) \si{\milli\tesla} pour celui de 2 \si{\micro\meter}, alors que le courant d'alimentation des bobines était le même.
