\section{Introduction}

L'effet Hall a été découvert par Edwin Hall en 1879 \cite{hall}. Ses application sont aujourd'hui multiples dans de nombreuses industries notamment grâce au développement des senseurs à effet Hall. Ils permettent de créer des détecteurs de position et de vitesse utilisés, par exemple, pour suivre et déclencher la séquence d'ignition d'un moteur à combustion, ou encore pour suivre la position d'un joystick. L'avantage de ces capteurs face à d'autres est qu'ils ne nécessitent pas d'ouverture ni de contact mécanique et sont donc par construction plus résilients. \cite{hall_applications}\\
Dans cette expérience l'effet Hall va être étudié sous différentes conditions pour plusieurs matériaux. Le but des manipulations sera d'obtenir leur constante de Hall et leur densité de charge afin de les caractériser et de connaitre leur comportement face à l'effet Hall. Dans un premier temps, deux échantillons du même semiconducteur (InP dopé au Si) d'épaisseurs différentes seront comparés. La deuxième étape de cette expérience sera de déterminer ces caractéristiques pour trois échantillons métaliques (argent, cuivre et bismuth). Finalement, l'hystérèse de la génération du champ magnétique avec les bobines de cuivre sera observée.
