\section{Introduction}

L'effet Hall a été découvert par Edwin Hall en 1879 \cite{hall}. Ses application aujourd'hui sont multiples, grace aux senseurs à effet Hall. Ils permettent de créer des détecteurs de position et vitesse très précis et sont utilisés par exemple pour suivre et permettre la séquence d'ignition d'un moteur à combustion, ou encore pour suivre la position d'un joystick. L'avantage de ces capteurs fâce à d'autres et qu'il ne nécessite pas d'ouverture ni de contact mécanique et est donc par construction plus résilient. \cite{hall_applications}\\
Dans cette expérience l'effet Hall va être observé sous différentes conditions dans différents matériaux, en mesurant leurs constante de Hall et leurs densités de charge. Dans un premier temps, deux échantillons du même semiconducteur (InP dopé au Si) d'épaisseurs différentes seront comparés, puis les caractéristiques de trois échantillons métaliques (argent, cuivre et bismuth) seront déterminés.
