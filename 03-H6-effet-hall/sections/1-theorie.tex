\section{Théorie}

\begin{minipage}{\textwidth}
    \begin{wrapfigure}{R}{0.5\textwidth}
        \centering
        \includegraphics[width=\linewidth]{figures/hall.png}
        \caption{Conducteur électrique de section \(S = a \cdot b\) placé dans un champ magnétique \(\vec{B}\) \cite{notice}}
        \label{fig:hall}
        \vspace*{1cm}
    \end{wrapfigure}

    Lorsqu'un conducteur ou un semiconducteur est placé dans un champ magnétique, les charges contenues dans ce matériau se déplacent selon une direction perpendiculaire au champ magnétique. Soit le repère carthésien \((\vec{e}_x,\vec{e}_y,\vec{e}_z)\) de la \autoref{fig:hall}. Avec le champ magnétique \(\vec{B}\) orienté selon \(\vec{e}_z\) et le déplacement de charges selon \(\vec{e}_x\), la force de Lorenz \(\vec{F}_L = q\vec{v} \times \vec{B}\) permet d'obtenir que le champ électrique \(\vec{E}\), dû à l'accumulation de charges d'un côté de l'échantillon, est selon \(-\vec{e}_y\). La différence de potentiel créé est donc selon cette direction et correspond à la tension de Hall \(V_H\):

    \begin{equation}
        V_H = -E_yb
    \end{equation}

    avec b la largeur de l'échantillon. L'effet Hall, proportionel à la tension au champ magnétique et au courant, est donné par l'équation

    \begin{equation}
        V_H = R_H j_x B_z b
        \label{eq:effet_hall}
    \end{equation}

    avec \(R_H\) la constante de Hall de l'échantillon en [\si{\meter\cubed\per\coulomb}], \(j_x\) la densité de courant traversant l'échantillon selon l'axe \(\vec{e}_x\) et \(B_z\) le champ magnétique selon \(\vec{e}_z\). \cite{notice}
\end{minipage}

Il est possible de montrer à l'aide de la loi d'Ohm, de l'effet Hall et de la densité de courant lié à la présence d'un champ magnétque que la constante de Hall \(R_H\) vaut

\begin{equation}
    R_H = \frac{1}{Nq}
    \label{eq:N}
\end{equation}

avec \(N\) la densité de porteurs de charge en [\si{\per\meter\cubed}] et \(q\) la charge du porteur de charge. Le signe de \(R_H\) permet donc de déterminer le signe des porteurs de charge.\\
En notant que la densité de charge est donnée par \(j = \frac{I}{S}\) où \(I\) est le courant et \(S\) la section de l'échantillon, l'\autoref{eq:effet_hall} donne:

\begin{equation}
    V_H = R_H \frac{IB}{a} \iff R_H = \frac{IB}{V_Ha}
    \label{eq:R_H}
\end{equation}

avec \(a\) l'épaisseur de l'echantillon comme indiqué à la \autoref{fig:hall}.
